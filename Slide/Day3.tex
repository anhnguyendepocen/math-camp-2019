\documentclass[pdflatex, 12pt]{beamer}
\usetheme{Boadilla}
\usefonttheme{professionalfonts}

\usepackage{graphicx}
\usepackage{color}
\usepackage{amsmath, amssymb}
\usepackage{bm}
%\usepackage{enumitem}
\usepackage{natbib}
\usepackage{url}
\usepackage{wasysym}
\usepackage{setspace}

%\setbeamerfont{title}{series=\bfseries}
%\setbeamerfont{frametitle}{series=\bfseries}

%\setbeamercolor{title}{fg=violet}
%\setbeamercolor{frametitle}{fg=violet}

%\setbeamercolor{palette primary}{fg=black, bg=violet!50}
%\setbeamercolor{palette secondary}{fg=violet, bg=white}
%\setbeamercolor{palette tertiary}{fg=black, bg=violet!70}

\setbeamertemplate{navigation symbols}{}

%\setbeamertemplate{itemize item}{\color{violet}$\bullet$}
%\setbeamertemplate{itemize subitem}{\color{violet}\scriptsize{$\blacktriangleright$}}
%\setbeamertemplate{itemize subsubitem}{\color{violet}$\star$}

\setbeamertemplate{itemize item}{$\bullet$}
\setbeamertemplate{itemize subitem}{\scriptsize{$\blacktriangleright$}}
\setbeamertemplate{itemize subsubitem}{$\star$}

\setbeamertemplate{enumerate items}[default]

\newcommand{\R}{\mathbb{R}}
\newcommand{\Q}{\mathbb{Q}}
\newcommand{\Z}{\mathbb{Z}}
\newcommand{\N}{\mathbb{N}}

\title[Math Camp]{Day 3: Calculus 2}
\author[Ikuma Ogura]{Ikuma Ogura}
\institute[Georgetown]{Ph.D. student, Department of Government, Georgetown University}
\date[August 21, 2019]{August 21, 2019}

\begin{document}

\begin{frame}
\frametitle{}
\titlepage
\end{frame}

\begin{frame}
\frametitle{Today}
\begin{itemize}
\item Today
 \begin{itemize}
 \item Partial derivative
 \item Integral
  \begin{itemize}
  \item Definition
  \item Calculation rules
  \item Calculation techniques
  \end{itemize}
 \end{itemize}
\end{itemize}
\end{frame}

\begin{frame}
\frametitle{Partial Derivative}
\begin{itemize}
\item What to do when there are more than one variables?
\vspace{0.4cm}
\item \textbf{Partial derivative} of a function with more than one variable is defined as the derivative with respect to one of those variables with others held constant.
\vspace{0.4cm}
\item Formally, the partial derivative of $f(x_1, x_2, \cdots, x_n)$ with respect to $x_i$ is defined as
 {\scriptsize
 \begin{equation} 
 \frac{\partial f}{\partial x_i} (x_1, x_2, \cdots, x_n) = \lim_{h \to 0} \frac{f(x_1, x_2, \cdots, x_i + h, \cdots, x_n) - f(x_1, x_2, \cdots, x_i, \cdots, x_n)}{h} \notag
 \end{equation}  
 }
\vspace{0.2cm}
\item As in the case of derivative of univariate functions, we can take higher-order derivatives. 
\end{itemize}
\end{frame}

\begin{frame}
\frametitle{Partial Derivative: Example}
\begin{itemize}
\item Let $f(x, y) = 3x^2y + 2y^3$. Then,
 \begin{eqnarray}
 \frac{\partial f}{\partial x}(x, y) &=& (3x^2y)' + (2y^3)' \notag \\
 &=& 3y \cdot (x^2)' = 6xy \notag 
 \end{eqnarray}
 and 
 \begin{eqnarray}
 \frac{\partial f}{\partial y}(x, y) &=& (3x^2y)' + (2y^3)' \notag \\
 &=& 3x^2 \cdot (y)' + 2 \cdot (y^3)' = 3x^2 + 6y^2 \notag 
 \end{eqnarray}
\end{itemize}
\end{frame}

\begin{frame}
\frametitle{Partial Derivative: Example (cont.)}
\begin{itemize}
\item We can take second second-order derivative as
 \begin{eqnarray}
 \frac{\partial^2 f}{\partial x^2}(x, y) &=& \frac{\partial}{\partial x} \Big(\frac{\partial}{\partial x} f(x, y)\Big) \notag \\
 &=& (6xy)' = 6y \cdot (x)' = 6y \notag 
 \end{eqnarray}
 and
 \begin{eqnarray}
 \frac{\partial^2 f}{\partial y^2}(x, y) &=& \frac{\partial}{\partial y} \Big(\frac{\partial}{\partial y} f(x, y)\Big) \notag \\
 &=& (3x^2)' + (6y^2)' = 6 \cdot (y^2)' = 12y \notag 
 \end{eqnarray}
\end{itemize}
\end{frame}

\begin{frame}
\frametitle{Partial Derivative: Example (cont.)}
\begin{itemize}
\item We can also take second-order \textbf{mixed derivative} as
 \begin{eqnarray}
 \frac{\partial^2 f}{\partial x \partial y}(x, y) &=& \frac{\partial}{\partial x} \Big(\frac{\partial}{\partial y} f(x, y)\Big) \notag \\
 &=& (3x^2)' + (6y^2)' = 3 \cdot (x^2) = 6x \notag 
 \end{eqnarray}
\vspace{0.2cm}
\item The order of differentiation does not matter:
 \begin{eqnarray}
 \frac{\partial^2 f}{\partial x \partial y}(x, y) &=& \frac{\partial}{\partial y} \Big(\frac{\partial}{\partial x} f(x, y)\Big) \notag \\
 &=& (6xy)' = 6x \cdot (y)' = 6x \notag 
 \end{eqnarray}
\end{itemize}
\end{frame}

\begin{frame}
\frametitle{Partial Derivative: Example (cont.)}
\begin{itemize}
\item Let $f(x, y) = 3x^3y^2 - 3xy^2 - \sqrt{y} + x$. Then calculate the following partial derivatives.
 \begin{enumerate}
 \item $\frac{\partial f}{\partial x} (x, y)$
 \vspace{0.1cm}
 \item $\frac{\partial f}{\partial y} (x, y)$
 \vspace{0.1cm}
 \item $\frac{\partial^2 f}{\partial x^2} (x, y)$
 \vspace{0.1cm}
 \item $\frac{\partial^2 f}{\partial y^2} (x, y)$
 \vspace{0.1cm}
 \item $\frac{\partial^2 f}{\partial x \partial y} (x, y)$
 \end{enumerate}
\end{itemize}
\end{frame}

\begin{frame}
\frametitle{Partial Derivative: Application}
\begin{itemize}
\item Same as the univariate case, partial derivative stand for the \textbf{rate of change} of a function.
\vspace{0.4cm}
\item Application examples
 \begin{itemize}
 \item \textbf{Marginal effects}
  \begin{itemize}
  \item How much does the value of $y$ change due to a one-unit change in $x$?
  \item e.g., $y = \alpha + \beta_1 x + \beta_2 z + \beta_3 xz$  
  \end{itemize}
 \item \textbf{Multivariate optimization}
 \end{itemize} 
\end{itemize}
\end{frame}

\begin{frame}
\frametitle{Partial Derivative: Visual Explanation}
\begin{itemize}
\item For a point on a graph of a function with more than one input, there are an infinite number of tangent lines.
\vspace{0.4cm}  
\item Then how do we determine the rate of change of the function?
\vspace{0.4cm}
\item Partial derivative of $y = f(x_1, x_2, \cdots, x_n)$ with regard to $x_i$ computes the rate of change by finding the slope of a tangent line parallel to the $x_{i}y$-plane.
\vspace{0.4cm}
\item Let's take a look at an example of a function with two independent variables.
\end{itemize}
\end{frame}

\begin{frame}
\frametitle{Partial Derivative: Visual Explanation (cont.)}
\begin{columns}
\begin{column}{0.5\textwidth}
\centering
\includegraphics[scale = 0.7]{fig3_11.pdf}
\end{column}
\begin{column}{0.5\textwidth}
\begin{itemize}
\item Let $z = f(x, y) = x^2 + xy + y^2$
\vspace{0.4cm}
\item $\frac{\partial f}{\partial x} (x, y) = 2x + y$ describes how slope of the tangent lines paralell to the $xz$-plane changes with values of $x$ and $y$
\vspace{0.4cm}
\item It is same as
 \begin{itemize}
 \item first slicing the graph at a specific value of $y = y_0$
 \item then finding the slope of a tangent line on the sliced plane
 \end{itemize}
\end{itemize}
\end{column}
\end{columns}
\end{frame}

\begin{frame}
\frametitle{Partial Derivative: Visual Explanation (cont.)}
\begin{columns}
\begin{column}{0.5\textwidth}
\centering
\includegraphics[scale = 0.7]{fig3_11.pdf}
\end{column}
\begin{column}{0.5\textwidth}
\centering
\only<1>{\includegraphics[scale = 0.5]{fig3_12.png}}
\only<2>{\includegraphics[scale = 0.5]{fig3_13.png}}
\only<3>{\includegraphics[scale = 0.5]{fig3_14.png}}
\end{column}
\end{columns}
\end{frame}

\begin{frame}
\frametitle{Multivariate Optimization}
\begin{itemize}
\item How to find local minima/maxima for functions with several inputs?
\vspace{0.4cm}
\item Let's again use an example of a function $f(x, y)$ with two independent variables.
\vspace{0.4cm}
\item At a local minimum/maximum, slope of the tangent lines on a $xz$-plane and a $yz$-place must be 0! 
\vspace{0.4cm}
\item Therefore, we need to solve the system of equations
 \begin{eqnarray}
 \begin{cases}
 \frac{\partial f}{\partial x} (x, y) = 0 \\
 \frac{\partial f}{\partial y} (x, y) = 0 \notag
 \end{cases}
 \end{eqnarray} 
\end{itemize}
\end{frame}

\begin{frame}
\frametitle{Multivariate Optimization (cont.)}
\begin{itemize}
\item By extending the logic, to find local minimum/maximum of a function $f(x_1, x_2, \cdots, x_n)$, we need to solve the system of equations
 {\normalsize
 \begin{eqnarray}
 \begin{cases}
 \frac{\partial f}{\partial x_1} (x_1, x_2, \cdots, x_n) = 0 \\
 \frac{\partial f}{\partial x_2} (x_1, x_2, \cdots, x_n) = 0 \\
 \ \ \vdots \\
 \frac{\partial f}{\partial x_i} (x_1, x_2, \cdots, x_n) = 0 \\
 \ \ \vdots \\
 \frac{\partial f}{\partial x_n} (x_1, x_2, \cdots, x_n) = 0 \notag
 \end{cases}
 \end{eqnarray}
 }
\end{itemize}
\end{frame}

\begin{frame}
\frametitle{Multivariate Optimization (cont.)}
\begin{itemize}
\item Second order condition?
 \begin{itemize}
 \item Distinguishing local maxima/minima is much harder in a multivariate case.
 \item We'll briefly touch on this issue tomorrow.
 \end{itemize}
\end{itemize}
\end{frame}

\begin{frame}
\frametitle{Multivariate Optimization: Example}
\begin{columns}
\begin{column}{0.5\textwidth}
\centering
\includegraphics[scale = 0.6]{fig3_15.pdf}
\end{column}
\begin{column}{0.5\textwidth}
\begin{itemize}
\item We often want to know the relationship between one variable ($x$) and another ($y$).
 \begin{itemize}
 \item e.g., GRE score and grades in graduate school
 \item Notation
  \begin{itemize}
  \item Total number of observations/points: $n$
  \item Denote each observation/point as $(x_1, y_1), (x_2, y_2),$ $\cdots, (x_n, y_n)$
  \end{itemize}
 \end{itemize}
\vspace{0.4cm}
\item Let's consider drawing a line that best fit the data.
 \begin{itemize}
 \item $y = a + bx$
 \end{itemize}
\end{itemize}
\end{column}
\end{columns}
\end{frame}

\begin{frame}
\frametitle{Multivariate Optimization: Example (cont.)}
\begin{columns}
\begin{column}{0.5\textwidth}
\centering
\includegraphics[scale = 0.6]{fig3_16.pdf}
\end{column}
\begin{column}{0.5\textwidth}
\begin{itemize}
\item \textbf{Ordinary Least Square (OLS)}: find a line which minimizes the sum of squared residuals
\vspace{0.4cm}
\item What is a residual?
 \begin{itemize}
 \item Distance between the line and each point.
 \end{itemize}
\vspace{0.4cm}
\item Residual for point $(x_i, y_i)$:
 \begin{itemize}
 \item A point on a line at $x = x_i$: $(x_i, a + bx_i)$
 \item Therefore, a residual for point $(x_i, y_i)$ is $y_i - (a + bx_i)$
 \end{itemize}
\end{itemize}
\end{column}
\end{columns}
\end{frame}

\begin{frame}
\frametitle{Multivariate Optimization: Example (cont.)}
\begin{columns}
\begin{column}{0.5\textwidth}
\centering
\includegraphics[scale = 0.6]{fig3_16.pdf}
\end{column}
\begin{column}{0.5\textwidth}
\begin{itemize}
\item Residual for observation $i$
 \begin{equation}
 y_i - (a + bx_i) \notag
 \end{equation}
\item Squared residual for observation $i$
 \begin{equation}
 \left\{y_i - (a + bx_i)\right\}^2 \notag
 \end{equation}
\item Sum of squared residuals
 \begin{equation}
 \sum_{i = 1}^{n} \left\{y_i - (a + bx_i)\right\}^2 \notag
 \end{equation}
\end{itemize}
\end{column}
\end{columns}
\end{frame}

\begin{frame}
\frametitle{Multivariate Optimization: Example (cont.)}
\begin{itemize}
\item Let's find the values of $a$ and $b$ which minimize the sum of squared residuals.
 \begin{equation}
 \operatorname*{argmax}_{a, b} \sum_{i = 1}^{n} \left\{y_i - (a + bx_i)\right\}^2 \notag
 \end{equation}
\item Following the discussion so far, we need to solve the system of equations
 \begin{eqnarray}
 \begin{cases}
 \frac{\partial}{\partial a} \sum_{i = 1}^{n} \left\{y_i - (a + bx_i)\right\}^2 = 0 \\
 \frac{\partial}{\partial b} \sum_{i = 1}^{n} \left\{y_i - (a + bx_i)\right\}^2 = 0 \notag
 \end{cases}
 \end{eqnarray}
\item NB: Here we treat $a$ and $b$ as variables and $x_i$s and $y_i$s and as constant!
\end{itemize}
\end{frame}

\begin{frame}
\frametitle{Multivariate Optimization: Example (cont.)}
\begin{itemize}
\item Let's compute the partial derivatives.
\end{itemize}
{\footnotesize
\begin{eqnarray}
&& \frac{\partial}{\partial a} \sum_{i = 1}^{n} \left\{y_i - (a + bx_i)\right\}^2 \notag \\
&=& \frac{\partial}{\partial a} \big(\left\{y_1 - (a + bx_1)\right\}^2 + \left\{y_2 - (a + bx_2)\right\}^2 \cdots + \left\{y_n - (a + bx_n)\right\}^2 \big) \notag \\
&=& \frac{\partial}{\partial a} \left\{y_1 - (a + bx_1)\right\}^2 + \frac{\partial}{\partial a} \left\{y_2 - (a + bx_2)\right\}^2 + \cdots + \frac{\partial}{\partial a} \left\{y_n - (a + bx_n)\right\}^2 \notag \\
&=& -2\left\{y_1 - (a + bx_1)\right\} - 2\left\{y_2 - (a + bx_2)\right\} - \cdots - 2\left\{y_n - (a + bx_n)\right\} \notag \\
&=& -2\big(\left\{y_1 - (a + bx_1)\right\} + \left\{y_2 - (a + bx_2)\right\} + \cdots + \left\{y_n - (a + bx_n)\right\}\big) \notag \\
&=& -2 \sum_{i = 1}^{n} \left\{y_i - (a + bx_i)\right\} \notag
\end{eqnarray}
}
\end{frame}

\begin{frame}
\frametitle{Multivariate Optimization: Example (cont.)}
and
{\footnotesize
\begin{eqnarray}
&& \frac{\partial}{\partial b} \sum_{i = 1}^{n} \left\{y_i - (a + bx_i)\right\}^2 \notag \\
&=& \frac{\partial}{\partial b} \big(\left\{y_1 - (a + bx_1)\right\}^2 + \left\{y_2 - (a + bx_2)\right\}^2 \cdots + \left\{y_n - (a + bx_n)\right\}^2 \big) \notag \\
&=& \frac{\partial}{\partial b} \left\{y_1 - (a + bx_1)\right\}^2 + \frac{\partial}{\partial b} \left\{y_2 - (a + bx_2)\right\}^2 + \cdots + \frac{\partial}{\partial b} \left\{y_n - (a + bx_n)\right\}^2 \notag \\
&=& -2x_1\left\{y_1 - (a + bx_1)\right\} - 2x_2\left\{y_2 - (a + bx_2)\right\} - \cdots - 2x_n\left\{y_n - (a + bx_n)\right\} \notag \\
&=& -2\big(x_1\left\{y_1 - (a + bx_1)\right\} + x_2\left\{y_2 - (a + bx_2)\right\} + \cdots + x_n\left\{y_n - (a + bx_n)\right\}\big) \notag \\
&=& -2 \sum_{i = 1}^{n} x_i\left\{y_i - (a + bx_i)\right\} \notag
\end{eqnarray}
}
\end{frame}

\begin{frame}
\frametitle{Multivariate Optimization: Example (cont.)}
\begin{itemize}
\item The system of equation
 \begin{eqnarray}
 \begin{cases}
 -2 \sum_{i = 1}^{n} \left\{y_i - (a + bx_i)\right\} = 0 \\
 -2 \sum_{i = 1}^{n} x_i\left\{y_i - (a + bx_i)\right\} = 0 \notag
 \end{cases}
 \end{eqnarray}
is called the \textbf{normal equation}. 
\vspace{0.4cm}
\item Rest is to solve the system of equations above!
 \begin{itemize}
 \item which you are asked to do in GOVT 701...
 \end{itemize}
\end{itemize}
\end{frame}

\begin{frame}
\frametitle{Integral}
\begin{itemize}
\item What is an integral?
 \begin{enumerate}
 \item \textbf{Antiderivative}
  \begin{itemize}
  \item inverse of a derivative
  \end{itemize}
 \item \textbf{Area under the curve}
 \end{enumerate}
\vspace{0.4cm}
\item \textbf{Fundamental theory of calculus} connects these two explanations.
\end{itemize}
\end{frame}

\begin{frame}
\frametitle{Indefinite Integral}
\begin{itemize}
\item \textbf{Antiderivative} or \textbf{indefinite integral} of a function $f(x)$, often denoted as $F(x)$, is a function whose derivative is equal to the riginal function $f(x)$. Formally,
 \begin{equation}
 F'(x) = f(x) \notag 
 \end{equation} 
\item Indefinite integral of $f(x)$ is also written as 
 \begin{equation}
 \int f(x) dx, \notag 
 \end{equation}
which is equal to $F(x)$. 
\end{itemize}
\end{frame}

\begin{frame}
\frametitle{Indefinite Integral (cont.)}
\begin{itemize}
\item Example: The function $F(x) = x^3$ is an antiderivative of $f(x) = 3x^2$, since $F'(x) = f(x)$.
\vspace{0.4cm}
\item However, this is not the only antiderivative.
 \begin{itemize}
 \item $F(x) = x^3 + 1$
 \item $F(x) = x^3 + 100$
 \item $F(x) = x^3 - 23$
 \item $F(x) = x^3 - 1108564$
 \item $\cdots$
 \end{itemize}
\vspace{0.4cm}
\item Since the derivative of a constant is 0, there are infinite number of antiderivatives!
\end{itemize}	
\end{frame}

\begin{frame}
\frametitle{Indefinite Integral (cont.)}
\begin{itemize}
\item To represent this situation, we add an arbitrary constant, $C$, known as the \emph{constant of integration}.
\vspace{0.4cm}
\item Example: antiderivative of $f(x) = 3x^2$ is 
 \begin{equation}
 F(x) = x^3 + C \notag
 \end{equation}
 \begin{itemize}
 \item All values of $F(x)$ can be obtained by changing the values of $C$.
 \end{itemize}
\end{itemize}	
\end{frame}

\begin{frame}
\frametitle{Definite Integral}
\begin{columns}
\begin{column}{0.5\textwidth}
\includegraphics[scale = 0.6]{fig3_1.pdf}
\end{column}
\begin{column}{0.5\textwidth}
\begin{itemize}
\item Suppose we want to find the area under the curve defined by a function $f(x)$ and some interval $x \in [a, b]$.
\vspace{0.4cm}
\item Let's approximinate the area with a seris of rectangles.
 \begin{itemize}
 \item First partition the interval into $n$ regions.
 \item Let the length of each subinterval be $\Delta x$
 \item Denote the mid-points of subintervals as $x_1, x_2, \cdots, x_n$ 
 \end{itemize}
\end{itemize}
\end{column}
\end{columns}
\end{frame}

\begin{frame}
\frametitle{Definite Integral (cont.)}
\begin{columns}
\begin{column}{0.5\textwidth}
\only<1>{\includegraphics[scale = 0.6]{fig3_2.pdf}}
\only<2>{\includegraphics[scale = 0.6]{fig3_3.pdf}}
\only<3>{\includegraphics[scale = 0.6]{fig3_4.pdf}}
\end{column}
\begin{column}{0.5\textwidth}
\begin{itemize}
\item Area of first rectangle: $f(x_1) \Delta x$
\vspace{0.4cm}
\item Area of second rectangle: $f(x_2) \Delta x$
\vspace{0.4cm}
\item $\cdots$
\vspace{0.4cm}
\item Add area of all the rectangles (\textbf{Riemann sum}):
 \begin{equation}
 \sum_{i = 1}^{n} f(x_i) \Delta x \notag
 \end{equation}
\end{itemize}
\end{column}
\end{columns}
\end{frame}

\begin{frame}
\frametitle{Definite Integral (cont.)}
\begin{columns}
\begin{column}{0.5\textwidth}
\only<1>{\includegraphics[scale = 0.6]{fig3_4.pdf}}
\only<2>{\includegraphics[scale = 0.6]{fig3_5.pdf}}
\only<3>{\includegraphics[scale = 0.6]{fig3_6.pdf}}
\end{column}
\begin{column}{0.5\textwidth}
\begin{itemize}
\item By making $\Delta x$ smaller and smaller (i.e., partitioning the interval $[a,b]$ into smaller regions)...
\vspace{0.4cm}
\item We can approximate the area closer and closer!
\vspace{0.4cm}
\item Formally stated, we can find the area under the curve by taking the limit of the Riemann sum as $\Delta x \to 0$:
 \begin{equation}
 \lim_{\Delta x \to 0} \sum_{i = 1}^{n} f(x_i) \Delta x \notag
 \end{equation}
\end{itemize}
\end{column}
\end{columns}
\end{frame}

\begin{frame}
\frametitle{Definite Integral (cont.)}
\begin{itemize}
\item \textbf{Riemann integral}/\textbf{Definite integral} of function $f(x)$ from $a$ to $b$ is defined as the limit of Riemann sum of $f(x)$ on that interval as $\Delta x \to 0$ and denoted as 
 \begin{equation}
 \int^b_a f(x)dx = \lim_{\Delta x \to 0} \sum_{i = 1}^{n} f(x_i) \Delta x \notag
 \end{equation}
\end{itemize}
\end{frame}

\begin{frame}
\frametitle{Fundamental Theorem of Calculus}
\begin{itemize}
\item (Second) \textbf{Fundamental Theorem of Calculus}: Let $f(x)$ be a function on an interval $[a, b]$, and $F(x)$ be an antiderivative of $f(x)$ on $[a, b]$. Then,
 \begin{equation}
 \int^b_a f(x)dx = F(b) - F(a) \notag
 \end{equation}
\item Therefore, we can calcuate the definite integral (i.e., area under the curve) from $b$ and $a$ by substracting the indefinite integral evaluated at $a$ from the indefinite integral evaluated at $b$! 
\end{itemize}
\end{frame}

\begin{frame}
\frametitle{Fundamental Theorem of Calculus (cont.)}
\begin{columns}
\begin{column}{0.5\textwidth}
\includegraphics[scale = 0.6]{fig3_7.pdf}
\end{column}
\begin{column}{0.5\textwidth}
\begin{itemize}
\item But why?
\vspace{0.4cm}
\item Let $S(b)$ be a function representing the area under the curve defined by the function $f(x)$ and interval $x \in [a, b]$.
\end{itemize}
\end{column}
\end{columns}
\end{frame}

\begin{frame}
\frametitle{Fundamental Theorem of Calculus (cont.)}
\begin{columns}
\begin{column}{0.5\textwidth}
\includegraphics[scale = 0.6]{fig3_8.pdf}
\end{column}
\begin{column}{0.5\textwidth}
\begin{itemize}
\item Let's consider how much the area changes if we move rightward from $b$ by $h$.
\vspace{0.4cm}
\item The area shown in blue can computed as $S(b + h) - S(b)$
\end{itemize}
\end{column}
\end{columns}
\end{frame}

\begin{frame}
\frametitle{Fundamental Theorem of Calculus (cont.)}
\begin{columns}
\begin{column}{0.5\textwidth}
\only<1>{\includegraphics[scale = 0.6]{fig3_9.pdf}}
\only<2>{\includegraphics[scale = 0.6]{fig3_10.pdf}}
\end{column}
\begin{column}{0.5\textwidth}
\begin{itemize}
\item Here let's assume that $f(x)$ is increasing around $b$.
\vspace{0.4cm}
\item Then, as we can see from the figure, $S(b + h) - S(b)$ is larger than $f(b)h$ and smaller than $f(b + h)h$. Dividing each by $h$, we get
 {\footnotesize
 \begin{equation}
 f(b) < \frac{S(b + h) - S(b)}{h} < f(b + h) \notag
 \end{equation}
 }
\end{itemize}
\end{column}
\end{columns}
\end{frame}

\begin{frame}
\frametitle{Fundamental Theorem of Calculus (cont.)}
\begin{itemize}
\item Take the limit of each as $h \to 0$.
 \begin{equation}
 \lim_{h \to 0} f(b) < \lim_{h \to 0} \frac{S(b + h) - S(b)}{h} < \lim_{h \to 0} f(b + h) \notag
 \end{equation}
\vspace{0.1cm}
\item Since $\lim_{h \to 0} f(b) = \lim_{h \to 0} f(b + h) = f(b)$, 
 \begin{equation}
 \lim_{h \to 0} \frac{S(b + h) - S(b)}{h} = f(b) \notag
 \end{equation}
\vspace{0.1cm}
\item From the definition of derivative, 
 \begin{equation}
 S'(b) = f(b) \notag
 \end{equation}
\end{itemize}
\end{frame}

\begin{frame}
\frametitle{Fundamental Theorem of Calculus (cont.)}
\begin{itemize}
\item Find the antiderivative of both:
 \begin{equation}
 S(b) = F(b) + C \notag
 \end{equation}
\vspace{0.1cm}
\item Since $S(a) = 0$, $C = -F(a)$. Therefore,
 \begin{equation}
 S(b) = F(b) - F(a) \notag
 \end{equation}
\vspace{0.1cm}
\item We can compute the area under the curve by calculating the differences in indefinite integrals!
\end{itemize}
\end{frame}

\begin{frame}
\frametitle{Rules of Integration}
\begin{itemize}
\item Common rules of integration
 \begin{enumerate}
 \item \textbf{Linearity}: $\int \left\{\alpha f(x) + \beta g(x) \right\}dx = \alpha \int f(x)dx + \beta \int g(x)dx$
 \vspace{0.1cm}
 \item \textbf{Reverse power rule}: $\int x^n dx = \frac{1}{n + 1} x^{n + 1} + C \ (n \neq -1)$
 \vspace{0.1cm}
 \item Exception to 2. when $n = -1$: $\int \frac{1}{x} dx = \log x + C$
 \vspace{0.1cm}
 \item \textbf{Exponential rule}: $\int a^x dx = \frac{a^x}{\log a} + C$
 \vspace{0.1cm}
 \item Special case of 4.: $\int e^x dx = e^x + C$
 \end{enumerate}
\end{itemize}
\end{frame}

\begin{frame}
\frametitle{Rules of Integration (cont.)}
\begin{itemize}
\item Definite integral $\int^b_a f(x) dx$:
 \begin{enumerate}
 \item Find the indefinite integral $F(x)$
 \item Calculate $F(b) - F(a)$
 \end{enumerate}
 \begin{itemize}
 \item We often write the second step as $\int^b_a f(x) dx = F(x)|^b_a$
 \end{itemize}
\vspace{0.4cm}
\item Common rules for definite integral
 \begin{enumerate}
 \item $\int^a_a f(x) dx = 0$
 \item $\int^b_a f(x) dx = -\int^a_b f(x) dx$
 \item $\int^b_a f(x) dx = \int_a^c f(x) dx + \int^b_c f(x) dx \ (c \in [a, b])$
 \item $\int^b_a \left\{\alpha f(x) + \beta g(x)\right\} dx = \alpha \int^b_a f(x) dx + \beta \int^b_a g(x) dx$
 \end{enumerate}
\end{itemize}
\end{frame}

\begin{frame}
\frametitle{Rules of Integration: Example}
\begin{itemize}
\item Find the antiderivative of $f(x) = 3x^3 - 4x^2 + x + 6$
\vspace{0.4cm}
 \begin{itemize}
 \item Answer:
  \begin{eqnarray}
  \int f(x) dx &=& \int (3x^3 - 4x^2 + x + 6)dx \notag \\
  &=& \int (3x^3) dx + \int (-4x^2) dx + \int (x)dx + \int (6)dx \notag \\
  &=& \frac{3}{4}x^4 - \frac{4}{3}x^3 + \frac{1}{2}x^2 + 6x + C \notag
  \end{eqnarray}
 \end{itemize}
\end{itemize}
\end{frame}

\begin{frame}
\frametitle{Rules of Integration: Example (cont.)}
\begin{itemize}
\item Calculate the definite integral $\int^4_{-1} (3x^2 + 2x - 1)dx$
\vspace{0.4cm}
 \begin{itemize}
 \item Answer:
  \begin{eqnarray}
  && \int^4_{-1} (3x^2 + 2x - 1)dx \notag \\
  &=& 3\int^4_{-1} x^2 dx + 2\int^4_{-1} x dx - \int^4_{-1} (1)dx \notag \\
  &=& 3 \cdot \frac{1}{3} x^3|^4_{-1} + 2 \cdot \frac{1}{2} x^2|^4_{-1} - x|^4_{-1} \notag \\
  &=& \left\{64 - (-1)\right\} + (16 - 1) - \left\{4 - (-1)\right\} \notag \\
  &=& 65 + 15 - 5 = 75. \notag
  \end{eqnarray}
 \end{itemize}
\end{itemize}
\end{frame}

\begin{frame}
\frametitle{Rules of Integration: Exercises}
\begin{enumerate}
\item Calculate the following integrals.
 \begin{enumerate}
 \item $\int (4x^5 + 2x^2 + 5)dx$
 \item $\int (\sqrt{x} + 2x^3)dx$
 \item $\int (3x^{\frac{3}{2}} + 4^{x})dx$
 \item $\int^2_{-2} (3x^2 + x)dx$
 \item $\int^4_2 (3x^2 + x + 5)dx$
 \item $\int^{16}_1 (5x^{\frac{3}{2}} - 2x^{-\frac{5}{4}})dx$
 \end{enumerate}
\vspace{0.4cm}
\item Find the area intersected by $f(x) = x^2 - 3x + 3$ and $g(x) = -x^2 + 5x + 13$.
\end{enumerate}
\end{frame}

\begin{frame}
\frametitle{Integration Techniques}
\begin{enumerate}
\item Integration by parts
\vspace{0.4cm}
\item Integration by substitution
\vspace{0.4cm}
\item L'Hopital's rule
\end{enumerate}
\end{frame}

\begin{frame}
\frametitle{Integration by Parts}
\begin{itemize}
\item If the integrand $h(x)$ can be represented as the product of two functions, $f(x)$ and $g'(x)$, then 
 \begin{equation}
 \int h(x) dx = \int f(x)g'(x) dx = f(x)g(x) - \int f'(x)g(x) dx \notag
 \end{equation} 
\item In the case of definite integral:
 \begin{equation}
 \int^b_a h(x) dx = \int^b_a f(x)g'(x) dx = f(x)g(x)|^b_a - \int^b_a f'(x)g(x) dx \notag
 \end{equation} 
\item This formula can easily be derived from the product rule of derivative calculation.
\end{itemize}
\end{frame}

\begin{frame}
\frametitle{Integration by Parts: Example}
\begin{itemize}
\item Question: find the indefinite integral of $\log(x)$.
\vspace{0.4cm}
 \begin{itemize}
 \item Answer: As $\log(x) = \log(x) \cdot (x)'$,
  \begin{eqnarray}
  \int \log(x) dx &=& \int \log(x) \cdot (x)' dx \notag \\
  &=& x\log(x) - \int \frac{1}{x} \cdot x dx \notag \\
  &=& x\log(x) - \int (1) dx \notag \\
  &=& x\log(x) - x + C. \notag
  \end{eqnarray}
 \end{itemize}
\end{itemize}
\end{frame}

\begin{frame}
\frametitle{Integration by Substitution}
\begin{itemize}
\item If we can find a function of $t(x)$ and write the integrand $f(x)$ as 
 \begin{equation}
 f(x) = f(t) \frac{dt}{dx}, \notag
 \end{equation}
we can apply the integration by substitution formula
 \begin{equation}
 \int f(x) dx = \int f(t) \frac{dt}{dx} dx = \int f(t) dt \notag
 \end{equation}
\item In the case of definite integral: if $t$ moves from $\alpha$ to $\beta$ when $x$ moves from $a$ to $b$, then
 \begin{equation}
 \int^b_a f(x) dx = \int^b_a f(t) \frac{dt}{dx} dx = \int^{\beta}_{\alpha} f(t) dt \notag
 \end{equation}
 \begin{itemize}
 \item Note that the interval of integration can change due to the variable transformation.
 \end{itemize}
\end{itemize}
\end{frame}

\begin{frame}
\frametitle{Integration by Substitution: Example}
\begin{itemize}
\item Question: find the indefinite integral of $f(x) = 18x^2 \sqrt[3]{6x^3 - 7}$
\vspace{0.4cm}
 \begin{itemize}
 \item Answer: Let $t = 6x^3 - 7$. Then, $\frac{dt}{dx} = 18x^2$. Therefore,
  \begin{eqnarray}
  \int f(x) dx &=& \int {\color{red}18x^2} \sqrt[3]{\color{blue} 6x^3 - 7} dx \notag \\
  &=& \int \sqrt[3]{\color{blue} t} {\color{red} \frac{dt}{dx}} dx \notag \\
  &=& \int t^{\frac{1}{3}} dt \notag \\
  &=& \frac{3}{4} t^{\frac{4}{3}} + C \notag \\
  &=& \frac{3}{4} (6x^3 - 7)^{\frac{4}{3}} + C \notag
  \end{eqnarray}
 \end{itemize}
\end{itemize}
\end{frame}

\begin{frame}
\frametitle{Integration by Substitution: Summary of Steps}
\begin{enumerate}
\item Identify some part of $f(x)$ which can be simplified by substituting in a single variable $t$ (which is a function of $x$)
\item Compute $\frac{dt}{dx}$, and reexpress $f(x)$ using $t$ and $\frac{dt}{dx}$
\item Solve the indefinite integral
\item (For indefinite integration) Substitute back in for $x$
\item (For definite integration) Determine the new interval of integration $[\alpha, \beta]$, and evaluate the antiderivative at the boundary points.
\end{enumerate}
\end{frame}

\begin{frame}
\frametitle{Integration by Substitution: Another Example}
\begin{itemize}
\item Question: find the definite integral $\int^1_0 x\sqrt{x^2 + 1}dx$
\vspace{0.4cm}
 \begin{itemize}
 \item Answer: Let $t = x^2 + 1$. Then, $\frac{dt}{dx} = 2x$. Also, when $x$ moves from $1$ to $0$, $t$ moves from $1$ to $2$. Therefore,
  \begin{eqnarray}
  && \int^1_0 {\color{red} x}\sqrt{\color{blue} x^2 + 1}dx \notag \\
  &=& \int^1_0 \sqrt{\color{blue} t} \cdot {\color{red} \frac{1}{2}\frac{dt}{dx}} dx \notag \\
  &=& \frac{1}{2} \int^2_1 \sqrt{t} dt \notag \\
  &=& \frac{1}{2} \cdot \frac{2}{3} t^{\frac{3}{2}}|^2_1 \notag \\
  &=& \frac{1}{3} (2\sqrt{2} - 1) \notag 
  \end{eqnarray}
 \end{itemize}
\end{itemize}
\end{frame}

\begin{frame}
\frametitle{L'Hopital's Rule}
\begin{itemize}
\item Let's calculate the definite integral $\int^{\infty}_0 xe^{-x} dx$.
 \begin{eqnarray}
 \int^{\infty}_0 xe^{-x} dx &=& \int^{\infty}_0 x(-e^{-x})' dx \notag \\
 &=& -(xe^{-x}|^{\infty}_0)  - \int^{\infty}_0 (x)' \cdot (-e^{-x}) dx \notag \\
 &=& -(xe^{-x}|^{\infty}_0)  + \int^{\infty}_0 e^{-x} dx \notag
 \end{eqnarray}
\item How can we evaluate $xe^{-x}$ at $x \to \infty$?
 \begin{itemize}
 \item $\lim_{x \to \infty} \frac{x}{e^x} = \frac{\infty}{\infty}$??
 \item In such cases, L'Hopital's rule help you circumvent the problem.
 \end{itemize}
\end{itemize}
\end{frame}

\begin{frame}
\frametitle{L'Hopital's Rule (cont.)}
\begin{itemize}
\item \textbf{L'Hopital's rule}: Let functions $f(x)$ and $g(x)$ be differentiable at an open interval close to $c$. If $\lim_{x \to c} f(x) = \lim_{x \to c} g(x) = 0$ or $\pm\infty$, and $\lim_{x \to c} \frac{f'(x)}{g'(x)}$ exists,
 \begin{equation}
 \lim_{x \to c} \frac{f(x)}{g(x)} = \lim_{x \to c} \frac{f'(x)}{g'(x)} \notag 
 \end{equation} 
\vspace{0.1cm}
\item Applying the L'Hopital's rule, we can see that
 \begin{equation}
 \lim_{x \to \infty} \frac{x}{e^x} = \lim_{x \to \infty} \frac{x'}{(e^x)'} = \lim_{x \to \infty} \frac{1}{e^x} = 0 \notag 
 \end{equation} 
\end{itemize}
\end{frame}

\begin{frame}
\frametitle{Integration Techniques: Exercises}
\begin{itemize}
\item Calculate the following integrals.
 \begin{enumerate}
 \item $\int \frac{1}{1 + \exp(-x)}$
 \vspace{0.1cm}
 \item $\int^3_2 x\exp(x^2)dx$
 \item $\int^2_1 x\log x dx$
 \item $\int^1_0 \log (1 + \sqrt{x})dx$ \ (Hint: set $t = 1 + sqrt{x}$ and perform integration by substitution)
 \end{enumerate}
\end{itemize}
\end{frame}

\begin{frame}
\frametitle{Tomorrow}
\begin{itemize}
\item Problem set 3 $\rightarrow$ review in the morning
\vspace{0.4cm}
\item Tomorrow
 \begin{itemize}
 \item Vector and matrix algebra
 \item Vector/matrix noations for multivariate calculus
 \item Geometric meanings of vector/matrix algebra
 \item Moore \& Siegel, Chapters 12, 13, 15.2.2, 15.2.4, \& 16.1
 \end{itemize}
\end{itemize}
\end{frame}

\end{document}
