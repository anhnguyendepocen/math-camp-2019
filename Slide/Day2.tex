\documentclass[pdflatex, 12pt]{beamer}
\usetheme{Boadilla}
\usefonttheme{professionalfonts}

\usepackage{graphicx}
\usepackage{color}
\usepackage{amsmath, amssymb}
\usepackage{bm}
%\usepackage{enumitem}
\usepackage{natbib}
\usepackage{url}
\usepackage{wasysym}
\usepackage{setspace}

%\setbeamerfont{title}{series=\bfseries}
%\setbeamerfont{frametitle}{series=\bfseries}

%\setbeamercolor{title}{fg=violet}
%\setbeamercolor{frametitle}{fg=violet}

%\setbeamercolor{palette primary}{fg=black, bg=violet!50}
%\setbeamercolor{palette secondary}{fg=violet, bg=white}
%\setbeamercolor{palette tertiary}{fg=black, bg=violet!70}

\setbeamertemplate{navigation symbols}{}

%\setbeamertemplate{itemize item}{\color{violet}$\bullet$}
%\setbeamertemplate{itemize subitem}{\color{violet}\scriptsize{$\blacktriangleright$}}
%\setbeamertemplate{itemize subsubitem}{\color{violet}$\star$}

\setbeamertemplate{itemize item}{$\bullet$}
\setbeamertemplate{itemize subitem}{\scriptsize{$\blacktriangleright$}}
\setbeamertemplate{itemize subsubitem}{$\star$}

\setbeamertemplate{enumerate items}[default]

\newcommand{\R}{\mathbb{R}}
\newcommand{\Q}{\mathbb{Q}}
\newcommand{\Z}{\mathbb{Z}}
\newcommand{\N}{\mathbb{N}}

\title[Math Camp]{Day 2: Calculus 1}
\author[Ikuma Ogura]{Ikuma Ogura}
\institute[Georgetown]{Ph.D. student, Department of Government, Georgetown University}
\date[August 20, 2019]{August 20, 2019}

\begin{document}

\begin{frame}
\frametitle{}
\titlepage
\end{frame}

\begin{frame}
\frametitle{Today}
\begin{itemize}
\item Calculus 1
 \begin{itemize}
 \item Limit
 \item Derivative
  \begin{itemize}
  \item Definition
  \item Calculation rules
  \end{itemize}
 \item Unconstrained optimization
 \item (Time permitting) Taylor series expansion/approximation
 \end{itemize}
\end{itemize}
\end{frame}

\begin{frame}
\frametitle{Limit}
\begin{itemize}
\item The \textbf{limit} of function $f(x)$ is the value that $f(x)$ approaches as $x$ approaches to some value (say $a$).
\vspace{0.4cm}
\item Notation: if $f(x)$ approaches to $b$ when $x$ approaches $a$, we write
 \begin{equation}
 \lim_{x \to a} f(x) = b \notag
 \end{equation}
 or
 \begin{equation}
 f(x) \rightarrow b \ \mathrm{as} \ x \rightarrow a \notag
 \end{equation}
\end{itemize}
\end{frame}

\begin{frame}
\frametitle{Limit (cont.)}
\begin{itemize}
\item How about just plug in the number!?
\vspace{0.4cm}
\item We need the concept limit because...
 \begin{itemize}
 \item We often want to think about the behavior of functions when it approaches to infinity/negative infinity
 \item We may want to evaluate the value that $f(x)$ where it is not defined
  \begin{itemize}
  \item e.g., what happens to $\log(x)$ at $x = 0$?
  \end{itemize}
 \item Functions can be discontinuous 
 \end{itemize}
\end{itemize}
\end{frame}

\begin{frame}
\frametitle{Limit (cont.)}
\begin{itemize}
\item Example: Let's compute 
 \begin{equation}
 \lim_{x \to 2} \frac{x^2 - 5x + 6}{x - 2} \notag
 \end{equation}
 \begin{itemize}
 \item Answer: Since
  \begin{equation}
  \frac{x^2 - 5x + 6}{x - 2} = \frac{(x - 2)(x - 3)}{x - 2} = x - 3 \ (x \neq 2), \notag
  \end{equation}
  we can see that
  \begin{equation}
  \lim_{x \to 2} \frac{x^2 - 5x + 6}{x - 2} = -1 \notag
  \end{equation}
  even if the function is not defined at $x = 2$.
 \end{itemize}
\end{itemize}
\end{frame}

\begin{frame}
\frametitle{Limit (cont.)}
\begin{columns}
\begin{column}{0.5\textwidth}
\centering
\includegraphics[scale = 0.6]{fig2_17.png}
\end{column}
\begin{column}{0.5\textwidth}
\begin{itemize}
\item Behavior of $f(x) = \frac{x^2 - 5x + 6}{x - 2}$ around $x = 2$
\end{itemize}
\begin{tabular}{l l}
\hline\hline
$x$ & $f(x)$ \\\hline
$1.99$ & $1.01$ \\
$1.999$ & $-1.001$ \\
$1.9999$ & $-1.0001$ \\
$1.99999$ & $-1.00001$ \\
$1.999999$ & $-1.000001$ \\
$2$ & Not defined \\
$2.000001$ & $-0.999999$ \\
$2.00001$ & $-0.99999$ \\
$2.0001$ & $-0.9999$ \\
$2.001$ & $-0.999$ \\
$2.01$ & $-0.99$ \\\hline
\end{tabular}
\end{column}
\end{columns}
\end{frame}

\begin{frame}
\frametitle{Limit: Continuity}
\begin{minipage}{0.49\textwidth}
$f(x)$ is not continuous
\centering
\includegraphics[scale = 0.45]{fig2_1.png}
\end{minipage}
\begin{minipage}{0.49\textwidth}
$f(x)$ is continuous
\centering
\includegraphics[scale = 0.45]{fig2_2.png}
\end{minipage}
\end{frame}

\begin{frame}
\frametitle{Limit: Continuity (cont.)}
\begin{itemize}
\item One-sided limit
 \begin{itemize}
 \item \textbf{Right-sided limit} is the limit when $x$ approaches $a$ from above (from the right)
 \item \textbf{Left-sided limit} is the limit when $x$ approaches $a$ from below (from the left)
 \end{itemize}
\vspace{0.4cm}
\item Notation: We denote the right-sided limit as
 \begin{equation}
 \lim_{x \to a^{+}} f(x) \ \mathrm{or} \ \lim_{x \downarrow a} f(x) \notag
 \end{equation}
and left-sided limit as
 \begin{equation}
 \lim_{x \to a^{-}} f(x) \ \mathrm{or} \ \lim_{x \uparrow a} f(x) \notag
 \end{equation}
\end{itemize}
\end{frame}

\begin{frame}
\frametitle{Limit: Continuity (cont.)}
\begin{itemize}
\item Function $f(x)$ is continuous at $a$ if 
 \begin{equation}
 \lim_{x \downarrow a} f(x) = \lim_{x \uparrow a} f(x) \notag
 \end{equation}
 and
 \begin{equation}
 \lim_{x \to a} f(x) = f(a) \notag
 \end{equation}
 \begin{itemize}
 \item If $\lim_{x \downarrow a} f(x) \neq \lim_{x \uparrow a} f(x)$, $\lim_{x \to a} f(x)$ is not defined.
 \end{itemize}
\end{itemize}
\end{frame}

\begin{frame}
\frametitle{Limit: Properties}
\begin{itemize}
\item Properties of limit operators
 \begin{enumerate}
 \item $\lim_{x \to a} \left\{\alpha f(x) + \beta g(x)\right\} = \alpha \lim_{x \to a} f(x) + \beta \lim_{x \to a} g(x)$
 \item $\lim_{x \to a} f(x)g(x) = \lim_{x \to a} f(x) \lim_{x \to a} g(x)$
 \item $\lim_{x \to a} \frac{f(x)}{g(x)} = \frac{\lim_{x \to a} f(x)}{\lim_{x \to a} g(x)} \ (\lim_{x \to a} g(x) \neq 0)$
 \end{enumerate}
\vspace{0.4cm}
\item Proerty 1. is called the \textbf{linearity}, and operators with this property is called the linear operators.
 \begin{itemize}
 \item e.g., $\sum$
 \end{itemize}
\end{itemize}
\end{frame}

\begin{frame}
\frametitle{Limit: Calculating a Limit of a Function}
\begin{itemize}
\item Tips
 \begin{itemize}
 \item Simply the function as much as possible before computing the limit
 \item Graphing the function is often helpful
 \end{itemize}
\end{itemize}
\end{frame}

\begin{frame}
\frametitle{Limit: Exercises}
\begin{itemize}
\item For the following functions at the specified values, please answer that (i) whether they are defined, (ii) whether the limit is defined, and (iii) if the limit is defined, its value.
 \begin{enumerate}
 \item $x^2$ at $x = 3$
 \item $3x^2 + 5x - 9$ at $x = 2$
 \item $|3x - 2|$ at $x = \frac{2}{3}$
 \vspace{0.1cm}
 \item $\frac{x^2 - 4x - 5}{x + 1}$ at $x = -1$
 \vspace{0.1cm}
 \item $\frac{1}{x}$ when $x \rightarrow \infty$
 \end{enumerate}
\end{itemize}
\end{frame}

\begin{frame}
\frametitle{Derivative: Introduction}
\begin{itemize}
\item We want to know how a function $f(x)$ is curved.
 \begin{itemize}
 \item Is it increasing/decreasing? How fast?
  \begin{itemize}
  \item What is \textbf{the rate of change}?
  \end{itemize}
 \item At which point does it start to increase/decrease?
 \end{itemize}
\vspace{0.4cm}
\item Let's examine the behavior of $f(x)$ at $x = x_0$.
\end{itemize}
\end{frame}

\begin{frame}
\frametitle{Derivative: Introduction (cont.)}
\begin{columns}
\begin{column}{0.5\textwidth}
\centering
\includegraphics[scale = 0.5]{fig2_6.png}
\end{column}
\begin{column}{0.5\textwidth}
\begin{itemize}
\item How do we know that $f(x)$ is increasing/decreasing at $x = x_0$?
\vspace{0.4cm}
\item Let's take a look at the slope of the tangent line.
\vspace{0.4cm}
\item Tangent line: a line that just ``touches'' the curve at $x = x_0$.
\vspace{0.4cm}
\item But how do we calculate the slope of the tangent line?
\end{itemize}
\end{column}
\end{columns}
\end{frame}

\begin{frame}
\frametitle{Derivative: Introduction (cont.)}
\begin{columns}
\begin{column}{0.5\textwidth}
\centering
\includegraphics[scale = 0.5]{fig2_3.png}
\end{column}
\begin{column}{0.5\textwidth}
\begin{itemize}
\item Think about approximating the tangent by a line connecting the two points on $f(x)$, $(x_0, f(x_0))$ and $(x_0 + h, f(x_0 + h))$
\vspace{0.4cm}
\item Slope of the line:
 \begin{eqnarray}
 && \frac{f(x_0 + h) - f(x_0)}{(x + h) - x} \notag \\
 &=& \frac{f(x_0 + h) - f(x_0)}{h} \notag 
 \end{eqnarray}
\end{itemize}
\end{column}
\end{columns}
\end{frame}

\begin{frame}
\frametitle{Derivative: Introduction (cont.)}
\begin{columns}
\begin{column}{0.5\textwidth}
\centering
\only<1>{\includegraphics[scale = 0.5]{fig2_3.png}}
\only<2>{\includegraphics[scale = 0.5]{fig2_4.png}}
\only<3>{\includegraphics[scale = 0.5]{fig2_5.png}}
\only<4>{\includegraphics[scale = 0.5]{fig2_6.png}}
\end{column}
\begin{column}{0.5\textwidth}
\begin{itemize}
\item Now let's make the increment $h$ smaller and smaller...
\vspace{0.4cm}
\item The line mathces the tangent line!
\end{itemize}
\end{column}
\end{columns}
\end{frame}

\begin{frame}
\frametitle{Derivative: Introduction (cont.)}
\begin{itemize}
\item We call the slope of the tangent line as \textbf{derivative}.
\vspace{0.4cm}
\item Using the limit notation we introduced earlier, the derivative of $f(x)$ at a point $x = x_0$ is denoted as
 \begin{equation}
 \lim_{h \to 0} \frac{f(x_0 + h) - f(x_0)}{h} \notag
 \end{equation}
\end{itemize}
\end{frame}

\begin{frame}
\frametitle{Derivative: Introduction (cont.)}
\begin{columns}
\begin{column}{0.5\textwidth}
\centering
\only<1>{\includegraphics[scale = 0.4]{fig2_7.png}}
\only<2>{\includegraphics[scale = 0.4]{fig2_8.png}}
\only<3>{\includegraphics[scale = 0.4]{fig2_9.png}}
\only<4>{\includegraphics[scale = 0.4]{fig2_10.png}}
\end{column}
\begin{column}{0.5\textwidth}
\begin{itemize}
\item The slope of the tangent line is different at different values of $x$.
\vspace{0.4cm}
\item We can also think about derivative as a function of $x$. 
\end{itemize}
\end{column}
\end{columns}
\end{frame}

\begin{frame}
\frametitle{Derivative: Introduction (cont.)}
\begin{itemize}
\item \textbf{Definition}: the derivative of $f(x)$ with regard to $x$ is defined as
 \begin{equation} 
 \lim_{h \to 0} \frac{f(x + h) - f(x)}{h} \notag
 \end{equation}
 \begin{itemize}
 \item By plugging in concrete numbers, we can get the slope of the curve (tangent) at specific points.
 \item We also say ``to take the derivative'' as ``\textbf{differentiate}'' 
 \end{itemize}
\vspace{0.4cm}
\item Notation: the derivative of $y = f(x)$ with regard to $x$ is denoted as
 \begin{equation}
 \frac{dy}{dx}\ \mathrm{or} \ \frac{df(x)}{dx}. \notag
 \end{equation}
If the variable we take the derivative is obvious, we also write as
 \begin{equation}
 f'(x) \notag
 \end{equation}  
\end{itemize}
\end{frame}

\begin{frame}
\frametitle{Derivative: Introduction (cont.)}
\begin{columns}
\begin{column}{0.5\textwidth}
\centering
\includegraphics[scale = 0.3]{fig2_18.png}
\end{column}
\begin{column}{0.5\textwidth}
\begin{itemize}
\item Not all the functions are differentiable.
 \begin{itemize}
 \item Differentiable: the derivative $\lim_{h \to 0} \frac{f(x + h) - f(x)}{h}$ exists
 \item e.g., $f(x) = |x|$
 \end{itemize}
\vspace{0.4cm}
\item Differentiability and continuity
 \begin{itemize}
 \item Differentiable functions are always continuous
 \item Continuity does not necessarily mean differentiability
  \begin{itemize}
  \item e.g., $f(x) = |x|$
  \end{itemize}
 \end{itemize}
\end{itemize}
\end{column}
\end{columns}
\end{frame}

\begin{frame}
\frametitle{Derivative: Exercises}
\begin{enumerate}
\item Following the definition introduced earlier, calculate the derivative of $f(x) = x^2 + 3x$.
\vspace{0.4cm}
\item Define $f(x) = x^3$
 \begin{enumerate}
 \item Following the defition introduced earlier, compute $f'(x)$.
 \item Calculate (a) $f'(2)$, (b) $f'(-1)$, and (c) $f'(4)$
 \end{enumerate}
\end{enumerate}
\end{frame}

\begin{frame}
\frametitle{Rules for Differentiation}
\begin{itemize}
\item It is cumbersome to calculate the derivatives using formal definition!
\vspace{0.4cm}
\item So we rely on rules of differentiation.
\vspace{0.4cm}
\item Rules of Differentiation
 \begin{enumerate}
 \item \textbf{Power Rule}: $(x^n)' = nx^{n - 1}$
 \item \textbf{Summation Rule}: $\left\{\alpha f(x) + \beta g(x)\right\}' = \alpha f'(x) + \beta g'(x)$
 \item \textbf{Product Rule}: $\left\{f(x)g(x)\right\}' = f'(x)g(x) + f(x)g'(x)$
 \item \textbf{Quotient Rule}: $\big(\frac{f(x)}{g(x)}\big)' = \frac{f'(x)g(x) - f(x)g'(x)}{\left\{g(x)\right\}^2}$
 \end{enumerate}
\end{itemize}
\end{frame}

\begin{frame}
\frametitle{Rules for Differentiation (cont.)}
\begin{itemize}
\item Example: Differentiate $f(x) = x^3 + x^2 - x + 8$
\vspace{0.4cm}
 \begin{itemize}
 \item Answer: 
  \begin{eqnarray}
  f'(x) &=& (x^3 + x^2 - x + 8)' \notag \\
  &=& (x^3)' + (x^2)' + (-x)' + (8)' \notag \\
  &=& 3x^2 + 2x - 1 \notag
  \end{eqnarray}
 \end{itemize}
\end{itemize}
\end{frame}

\begin{frame}
\frametitle{Rules for Differentiation (cont.)}
\begin{itemize}
\item Example: Differentiate $f(x) = (x - 2)(x^2 + 3x + 1)$
\vspace{0.4cm}
 \begin{itemize}
 \item Answer: 
 \begin{eqnarray}
 f'(x) &=& \left\{(x - 2)(x^2 + 3x + 1)\right\}' \notag \\
 &=& (x - 2)'(x^2 + 3x + 1) + (x - 2)(x^2 + 3x + 1)' \notag \\
 &=& (x^2 + 3x + 1) + (x - 2)(2x + 3) \notag \\
 &=& (x^2 + 3x + 1) + (2x^2 + 3x - 4x - 6) \notag \\
 &=& 3x^2 + 2x - 5 \notag
 \end{eqnarray}
 \end{itemize}
\end{itemize}
\end{frame}	

\begin{frame}
\frametitle{Rules for Differentiation (cont.)}
\begin{itemize}
\item Example: Differentiate $f(x) = \frac{2x + 5}{3x^2}$
\vspace{0.4cm}
 \begin{itemize}
 \item Answer: 
 \begin{eqnarray}
 f'(x) &=& \Big(\frac{2x + 5}{3x^2}\Big)' \notag \\
 &=& \frac{(2x + 5)' \cdot (3x^2) - (2x + 5) \cdot (3x^2)'}{(3x^2)^2} \notag \\
 &=& \frac{(2) \cdot (3x^2) - (2x + 5) \cdot (6x)}{9x^4} \notag \\
 &=& \frac{6x^2 - 12x^2 - 30x}{9x^4} \notag \\
 &=& -\frac{2(x + 5)}{3x^3} \notag
 \end{eqnarray}
 \end{itemize}	
\end{itemize}
\end{frame}	

\begin{frame}
\frametitle{Rules for Differentiation: Exponents and Logs}
\begin{itemize}
\item Rules related with exponential and log functions
 \begin{enumerate}
 \item $(e^x)' = e^x$
 \item $(\log x)' = \frac{1}{x}$
 \item $(a^x)' = a^x \log x$
 \end{enumerate}
\end{itemize}
\end{frame}

\begin{frame}
\frametitle{Rules for Differentiation: Exponents and Logs (cont.)}
\begin{itemize}
\item Example: Differentiate $f(x) = 2x^2 e^x$
\vspace{0.4cm}
 \begin{itemize}
 \item Answer: 
 \begin{eqnarray}
 f'(x) &=& (2x^2 e^x)' \notag \\
 &=& (2x^2)' \cdot (e^x) + (2x^2) \cdot (e^x)' \notag \\
 &=& (4x) \cdot (e^x) + (2x^2) \cdot (e^x) \notag \\
 &=& (2x^2 + 4x) e^x \notag \\
 &=& 2x(x + 2) e^x \notag
 \end{eqnarray}
 \end{itemize}
\end{itemize}
\end{frame}

\begin{frame}
\frametitle{Rules for Differentiation: Chain Rule}
\begin{itemize}
\item Chain rule: used to differentiate composite functions
\vspace{0.4cm}
\item \textbf{Composite function} is a function whose input is the output of another function, denoted as 
 \begin{equation}
 h(x) = f(g(x)) = (f \circ g)(x) \notag
 \end{equation}
 \begin{itemize}
 \item Note that the range of the inner function (i.e., $g(x)$) must be contained in the domain of the outer function (i.e., $f(x)$).
 \item $(f \circ g)(x)$ and $(g \circ f)(x)$ are generally different.
 \item e.g., Let $f(x) = x^2$ and $g(x) = \log x$. Then,
  \begin{eqnarray}
  (f \circ g)(x) &=& (\log x)^2 \notag \\
  (g \circ f)(x) &=& \log x^2 \notag
  \end{eqnarray}
 \end{itemize}
\end{itemize}
\end{frame}

\begin{frame}
\frametitle{Rules for Differentiation: Chain Rule (cont.)}
\begin{itemize}
\item \textbf{Chain Rule}: Derivative of $h(x) = f(g(x))$ with respect to $x$ is
 \begin{equation}
 h(x)' = f'(g(x)) \cdot g'(x) \notag  
 \end{equation}
 \begin{itemize}
 \item Chain rule is also denoted as 
  \begin{equation}
  \frac{dh(x)}{dx} = \frac{dh(x)}{dg(x)} \frac{dg(x)}{dx} \notag
  \end{equation}
 \item \textit{The derivative of a composite function is the derivative of outer times the derivative of inner}
 \end{itemize}
\end{itemize}
\end{frame}

\begin{frame}
\frametitle{Rules for Differentiation: Chain Rule (cont.)}
\begin{itemize}
\item Example: Differentiate $h(x) = (\log x)^2$
\vspace{0.4cm}
 \begin{itemize}
 \item Answer: 
  \begin{eqnarray}
  h'(x) &=& \left\{(\log x)^2\right\}' \notag \\
  &=& \frac{d(\log x)^2}{d\log x} \frac{d\log x}{dx} \notag \\
  &=& 2\log x \cdot \frac{1}{x} \notag \\
  &=& \frac{2\log x}{x} \notag 
  \end{eqnarray}
 \end{itemize}
\end{itemize}
\end{frame}

\begin{frame}
\frametitle{Rules for Differentiation: Chain Rule (cont.)}
\begin{itemize}
\item Example: Differentiate $h(x) = \log x^2$
\vspace{0.4cm}
 \begin{itemize}
 \item Answer: 
  \begin{eqnarray}
  h'(x) &=& (\log x^2)' \notag \\
  &=& \frac{d\log x^2}{dx^2} \frac{dx^2}{dx} \notag \\
  &=& \frac{1}{x^2} \cdot (2x) \notag \\
  &=& \frac{2}{x} \notag
  \end{eqnarray}
 \end{itemize}
\end{itemize}
\end{frame}

\begin{frame}
\frametitle{Rules for Differentiation: Exercises}
\begin{itemize}
\item Differentiate the following functions with respect to $x$.
 \begin{enumerate}
 \item $4x^3 + \frac{1}{3} x^2 + 2x + 7$
 \item $(2x + 3)(x^2 - 13)$
 \vspace{0.1cm}
 \item $\frac{1}{\log x}$
 \vspace{0.1cm}
 \item $(\sqrt{x} + 3)(x^3 - x^2 + 1)$
 \item $(2x + 5)^3$
 \item $\exp(x^2)$
 \vspace{0.1cm}
 \item $\frac{1}{1 + \exp(-x)}$
 \vspace{0.1cm}
 \item $(x + 3)(x^2 - 3x + 9)$
 \item $\sqrt{x^2 + 1}$
 \item $\log e^x$
 \end{enumerate}
\end{itemize}
\end{frame}

\begin{frame}
\frametitle{Optimization}
\begin{itemize}
\item One major application of diffenretial calculus
\vspace{0.4cm}
\item When does the function takes the largest/smallest value?
 \begin{itemize}
 \item Many applications in formal modeling and statistics!
 \end{itemize}
\vspace{0.4cm}
\item ``Unconstrained'' optimization
 \begin{itemize}
 \item Find the value of inputs which maximizes/minimizes a function.
 \item Constrained optimization: Find the value of inputs which maximizes/minimizes a function under some constraints
  \begin{itemize}
  \item e.g., Find the value of $(x, y)$ which minimizes $f(x, y)$ under the constraint that $x + y \leq 5$.
  \end{itemize}
 \end{itemize}
\end{itemize}
\end{frame}

\begin{frame}
\frametitle{Optimization: Terms and Notations}
\begin{itemize}
\item \textbf{Objective function}: a function we want to maximize/minimize
\vspace{0.4cm}
\item The values of the variable $x$ which maximize the function $f(x)$ is denoted as 
 \begin{equation}
 \operatorname*{argmax}_x f(x) \notag
 \end{equation}
Similarly, the values of the variable $x$ which minimize the function $f(x)$ is denoted as
 \begin{equation}
 \operatorname*{argmin}_x f(x) \notag
 \end{equation}
\end{itemize}
\end{frame}

\begin{frame}
\frametitle{Optimization: Terms and Notations (cont.)}
\begin{itemize}
\item \textbf{Extrema} of a function are any points where the value of a function is the largest (\textbf{maxima}) or smallest (\textbf{minima}).
\vspace{0.4cm}
\item Global v. Local
 \begin{itemize}
 \item A point $(x_0, f(x_0))$ is a \textbf{global maximum} if $f(x_0) \geq f(x)$ for all $x$ in the domain.
 \item A point $(x_0, f(x_0))$ is a \textbf{global minimum} if $f(x_0) \leq f(x)$ for all $x$ in the domain.
 \item A point $(x_0, f(x_0))$ is a \textbf{local maximum} if $f(x_0) \geq f(x)$ for all $x$ within some open interval containing $x_0$.
 \item A point $(x_0, f(x_0))$ is a \textbf{local minimum} if $f(x_0) \leq f(x)$ for all $x$ within some open interval containing $x_0$.
 \end{itemize}
\end{itemize}
\end{frame}

\begin{frame}
\frametitle{Optimization: Terms and Notations (cont.)}
\begin{columns}
\begin{column}{0.5\textwidth}
\centering
\includegraphics[scale = 0.4]{fig2_12.png}
\end{column}
\begin{column}{0.5\textwidth}
\begin{itemize}
\item The blue point is a local maximum, and the red is a local minimum.
\vspace{0.4cm}
\item Whether they are also the global maximum/minimum depends on the domain of the function. 
\end{itemize}
\end{column}
\end{columns}
\end{frame}

\begin{frame}
\frametitle{Optimization: First Order Condition}
\begin{columns}
\begin{column}{0.5\textwidth}
\includegraphics[scale = 0.4]{fig2_29.png}
\end{column}
\begin{column}{0.5\textwidth}
\begin{itemize}
\item Let's find the local maxima of a function ($x^3 - 4x^2 + 3x + 1$) graphed on the left.
\vspace{0.4cm}
\item Because the local maxima are the points where the curve stops sloping upward and it starts sloping downward...
\end{itemize}
\end{column}
\end{columns}
\end{frame}

\begin{frame}
\frametitle{Optimization: First Order Condition (cont.)}
\begin{columns}
\begin{column}{0.5\textwidth}
\includegraphics[scale = 0.4]{fig2_19.png}
\end{column}
\begin{column}{0.5\textwidth}
\begin{itemize}
\item Local maxima are the points where the slope of the tangent line equals to zero!
\vspace{0.4cm}
\item \textbf{First Order Condition}: To find (local) extrema of $f(x)$, we need to look at points ($x_0, f(x_0)$) where
 \begin{equation}
 f'(x_0) = 0 \notag
 \end{equation}
\end{itemize}
\end{column}
\end{columns}
\end{frame}

\begin{frame}
\frametitle{Optimization: Second Order Condition}
\begin{columns}
\begin{column}{0.5\textwidth}
\includegraphics[scale = 0.4]{fig2_20.png}
\end{column}
\begin{column}{0.5\textwidth}
\begin{itemize}
\item First order condition cannot distinguish the maxima and minima, as the slope of the tangent line is also zero for the latter!
\vspace{0.4cm}
\item Then how should we do?
\end{itemize}
\end{column}
\end{columns}
\end{frame}

\begin{frame}
\frametitle{Optimization: Second Order Condition (cont.)}
\begin{columns}
\begin{column}{0.5\textwidth}
\only<1>{\includegraphics[scale = 0.4]{fig2_21.png}}
\only<2>{\includegraphics[scale = 0.4]{fig2_22.png}}
\only<3>{\includegraphics[scale = 0.4]{fig2_19.png}}
\only<4>{\includegraphics[scale = 0.4]{fig2_23.png}}
\only<5>{\includegraphics[scale = 0.4]{fig2_24.png}}
\end{column}
\begin{column}{0.5\textwidth}
\begin{itemize}
\item Let's take a look at the slope of the tangent line (= values of the derivatives) around the local maxima.
\vspace{0.4cm}
\item The slope of the tangent line is \textbf{decreasing} as $x$ gets larger!
\end{itemize}
\end{column}
\end{columns}
\end{frame}

\begin{frame}
\frametitle{Optimization: Second Order Condition (cont.)}
\begin{columns}
\begin{column}{0.5\textwidth}
\only<1>{\includegraphics[scale = 0.4]{fig2_25.png}}
\only<2>{\includegraphics[scale = 0.4]{fig2_26.png}}
\only<3>{\includegraphics[scale = 0.4]{fig2_20.png}}
\only<4>{\includegraphics[scale = 0.4]{fig2_27.png}}
\only<5>{\includegraphics[scale = 0.4]{fig2_28.png}}
\end{column}
\begin{column}{0.5\textwidth}
\begin{itemize}
\item Let's examine how the slope of the tangent line changes around the local minima.
\vspace{0.4cm}
\item The slope of the tangent line is \textbf{increasing} as $x$ gets larger!
\end{itemize}
\end{column}
\end{columns}
\end{frame}

\begin{frame}
\frametitle{Optimization: Second Order Condition (cont.)}
\begin{itemize}
\item How can we describe the change in the slope of the tangent line?
\vspace{0.4cm}
\item Because derivative describes the rate of change of a function ... $\rightarrow$ how about differentiate the function once more?
\vspace{0.4cm}
\item \textbf{Second derivative} of $f(x)$ is the derivative of $f'(x)$, which is often denoted as 
 \begin{equation}
 f''(x)\ \mathrm{or} \ \frac{d^2 f(x)}{dx^2} \notag 
 \end{equation}
\vspace{0.2cm}
\item By extention, we can also think about n-th derivative of $f(x)$, denoted as $f^{(n)} (x)$ or $\frac{d^n f(x)}{dx^n}$, by taking the derivative $n$ times.
 \begin{itemize}
 \item which describes the rate of change of rate of change of ...
 \end{itemize} 
\end{itemize}
\end{frame}

\begin{frame}
\frametitle{Optimization: Second Order Condition (cont.)}
\begin{itemize}
\item Convex v. Concave
 \begin{itemize}
 \item Function $f(x)$ is \textbf{convex} on an interval if $f''(x) > 0$ for all $x$ in that interval.
  \begin{itemize}
  \item If $f(x)$ is convex, $f(\alpha x_1 + (1 - \alpha) x_2) \leq \alpha f(x_1) + (1 - \alpha)f(x_2) \ (\alpha \in [0, 1])$ for all $x_1, x_2$ in that interval.
  \end{itemize}
 \item Function $f(x)$ is \textbf{concave} on an interval if $f''(x) < 0$ for all $x$ in that interval.
  \begin{itemize}
  \item If $f(x)$ is concave, $f(\alpha x_1 + (1 - \alpha) x_2) \geq \alpha f(x_1) + (1 - \alpha)f(x_2) \ (\alpha \in [0, 1])$ for all $x_1, x_2$ in that interval.
  \end{itemize}
 \end{itemize}
\end{itemize}
\end{frame}

\begin{frame}
\frametitle{Optimization: Second Order Condition (cont.)}
\begin{itemize}
\item Examples: convex (left column), concave (right column)
\end{itemize}
\begin{minipage}{0.49\textwidth}
\centering
\includegraphics[scale = 0.32]{fig2_13.png}
\end{minipage}
\begin{minipage}{0.49\textwidth}
\centering
\includegraphics[scale = 0.32]{fig2_15.png}
\end{minipage}
\begin{minipage}{0.49\textwidth}
\vspace{0.4cm}
\centering
\includegraphics[scale = 0.32]{fig2_14.png}
\end{minipage}
\begin{minipage}{0.49\textwidth}
\vspace{0.4cm}
\centering
\includegraphics[scale = 0.32]{fig2_16.png}
\end{minipage}
\end{frame}

\begin{frame}
\frametitle{Optimization: Second Order Condition (cont.)}
\begin{itemize}
\item \textbf{Second order condition}: when $f'(x_0) = 0$,
 \begin{itemize}
 \item if $f''(x_0) < 0$ (i.e., $f(x)$ is concave in an interval containing $x_0$), the point $(x_0, f(x_0))$ is the \textbf{local maximum}
 \item if $f''(x_0) > 0$ (i.e., $f(x)$ is convex in an interval containing $x_0$), the point $(x_0, f(x_0))$ is the \textbf{local minimum}
 \end{itemize}
\vspace{0.4cm}
\item When $f''(x_0) = 0$?
 \begin{itemize}
 \item e.g., $f(x) = x^3$
 \item Need to look at higher order derivatives
 \item (You rarely encounter these cases. No worries!)
 \end{itemize}
\vspace{0.4cm}
\item To see whether $x_0$ is also a global maximum/minimum,
 \begin{itemize}
 \item if the domain is bounded, check the values of $f(x)$ at the boundaries
 \item examine the signs of $f'(x)$ (and $f''(x)$) $\rightarrow$ detect the shape of the curve in the entire domain
 \end{itemize}
\end{itemize}
\end{frame}

\begin{frame}
\frametitle{Optimization: Example}
\begin{itemize}
\item Find all the extrema (local and global) of $f(x) = x^3 - x^2 \ (x \in [-1, 1])$, and state whether each extremum is a minimum or maximum and whether each is only local or global on that domain.
 \begin{itemize}
 \item Answer: First calculate $f'(x)$ and set it to 0,
  \begin{eqnarray}
  f'(x) = 3x^2 - 2x = x(3x - 2) &=& 0 \notag \\
  &\Rightarrow& x = 0, \frac{3}{2} \notag
  \end{eqnarray}
 Then compute $f''(x)$ and evaluate these points.
  \begin{eqnarray}
  && f''(x) = 6x - 2 \notag \\
  &\Rightarrow& f''(0) < 0, f''(\frac{3}{2}) > 0 \notag
  \end{eqnarray}
 Finally evalute $f(x)$ at these points and boundary points, revealing that $(-1, -2)$ is the global minimum, $(\frac{3}{2}, -\frac{4}{27})$ is the local minimum, $(0 ,0 )$ and $(1, 0)$ are the global maxima.
 \end{itemize}
\end{itemize}
\end{frame}

\begin{frame}
\frametitle{Optimization: Exercises}
\begin{itemize}
\item Find all the extrema (local and global) of the following functions on the specified domains, and state whether each extremum is a minimum or maximum and whether each is only local or global on that domain.
 \begin{enumerate}
 \item $f(x) = x^3 - x + 1, \ \ (x \in [0, 1])$
 \item $f(x) = x^3 - 3x, \ \ (x \in [-2, 2])$
 \end{enumerate}
\end{itemize}
\end{frame}

\begin{frame}
\frametitle{Taylor Series Expansion/Approximation}

\end{frame}

\end{document}
