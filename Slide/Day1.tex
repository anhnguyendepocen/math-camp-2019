\documentclass[pdflatex, 12pt]{beamer}
\usetheme{Boadilla}
\usefonttheme{professionalfonts}

\usepackage{graphicx}
\usepackage{color}
\usepackage{amsmath, amssymb}
\usepackage{bm}
%\usepackage{enumitem}
\usepackage{natbib}
\usepackage{url}
\usepackage{wasysym}
\usepackage{setspace}

%\setbeamerfont{title}{series=\bfseries}
%\setbeamerfont{frametitle}{series=\bfseries}

%\setbeamercolor{title}{fg=violet}
%\setbeamercolor{frametitle}{fg=violet}

%\setbeamercolor{palette primary}{fg=black, bg=violet!50}
%\setbeamercolor{palette secondary}{fg=violet, bg=white}
%\setbeamercolor{palette tertiary}{fg=black, bg=violet!70}

\setbeamertemplate{navigation symbols}{}

%\setbeamertemplate{itemize item}{\color{violet}$\bullet$}
%\setbeamertemplate{itemize subitem}{\color{violet}\scriptsize{$\blacktriangleright$}}
%\setbeamertemplate{itemize subsubitem}{\color{violet}$\star$}

\setbeamertemplate{itemize item}{$\bullet$}
\setbeamertemplate{itemize subitem}{\scriptsize{$\blacktriangleright$}}
\setbeamertemplate{itemize subsubitem}{$\star$}

\setbeamertemplate{enumerate items}[default]

\newcommand{\R}{\mathbb{R}}
\newcommand{\Q}{\mathbb{Q}}
\newcommand{\Z}{\mathbb{Z}}
\newcommand{\N}{\mathbb{N}}

\title[Math Camp]{Day 1: Basics}
\author[Ikuma Ogura]{Ikuma Ogura}
\institute[Georgetown]{Ph.D. student, Department of Government, Georgetown University}
\date[August 19, 2019]{August 19, 2019}

\begin{document}

\begin{frame}
\frametitle{}
\titlepage
\end{frame}

\begin{frame}
\frametitle{Why Do I Need to Study Mathematics?}
\begin{itemize}
\item Statistical analysis/Formal modeling
 \begin{itemize}
 \item Standard tools for political science research
 \item Applied field of mathematics
 \end{itemize}
\vspace{0.4cm}
\item We study mathematics to...
 \begin{itemize}
 \item read textbooks
 \item read articles using statsitics/formal modeling
 \end{itemize}
\end{itemize}
\end{frame}

\begin{frame}
\frametitle{\emph{Real Stats}}
\centering
\includegraphics[scale = 0.7]{fig1_bailey.png}
\end{frame}

\begin{frame}
\frametitle{\emph{Game Theory for Applied Economists}}
\centering
\includegraphics[scale = 0.5]{fig1_gibbons.png}
\end{frame}

\begin{frame}
\frametitle{\emph{Elements of Statistical Learning}}
\centering
\includegraphics[scale = 0.5]{fig1_esl.png}
\end{frame}

\begin{frame}
\frametitle{No Worries!}
\centering
\only<1>{\includegraphics[scale = 0.7]{fig1_bailey_1.png}}
\only<2>{\includegraphics[scale = 0.5]{fig1_gibbons_1.png}}
\only<3>{\includegraphics[scale = 0.5]{fig1_esl_1.png}}
\end{frame}

\begin{frame}
\frametitle{Note}
\begin{itemize}
\item The mathematics you need depends on what kinds of research you want to do.
\vspace{0.4cm}
\item This lecture will cover basic mathematics that you're likely to encounter
\end{itemize}
\end{frame}

\begin{frame}
\frametitle{Do I Need to Memorize Formulas?}
\begin{itemize}
\item NO!
\vspace{0.4cm}
\item Rather focus on understanding the meanings of the formulas and when/why/how you should use them.
\end{itemize}
\end{frame}

\begin{frame}
\frametitle{Today}
\begin{itemize}
\item Basics
 \begin{itemize}
 \item Algebra review
 \item Set
 \item Variable
 \item Fuctions
  \begin{itemize}
  \item Graphing equations/inequalities
  \item Exponential and log functions
  \end{itemize}
 \item Summation \& product operators
 \end{itemize}
\vspace{0.4cm}
\item (Today's lecture is a bit all over the place...)
\end{itemize}
\end{frame}

\begin{frame}
\frametitle{Algebra Review: Order of Calculation}
\begin{itemize}
\item Perform the following calculations.
 \begin{enumerate}
 \item $3 + 5 \times 2 - 17 = $
 \item $5 \times (7 - 4) + (4 - 2) = $
 \item $(8 - 3) - (9 - 13) \times 4 = $
 \end{enumerate}
\end{itemize}
\end{frame}

\begin{frame}
\frametitle{Algebra Review: Fraction}
\begin{itemize}
\item Perform the following calculations.
 \begin{enumerate}
 \item $\frac{2}{3} - \frac{1}{5} = $
 \vspace{0.1cm}
 \item $\frac{4}{5} \times \frac{2}{3} = $
 \vspace{0.1cm}
 \item $\frac{2}{5} \div \frac{3}{8} = $
 \vspace{0.1cm}
 \item $\frac{\frac{2}{7}}{\frac{3}{4}} = $
 \end{enumerate}
\end{itemize}
\end{frame}

\begin{frame}
\frametitle{Algebra Review: Power, Root, Factorial, \& Absolute value}
\begin{itemize}
\item Perform the following calculations.
 \begin{enumerate}
 \item $2^3 \times 4^2 = $
 \item $(2^3)^2 = $
 \item $3^{-1} = $
 \item $\sqrt[3]{8} = $
 \item $5! = $
 \item $|-3| = $
 \end{enumerate}
\end{itemize}
\end{frame}

\begin{frame}
\frametitle{Set}
\begin{itemize}
\item \textbf{Set}: a collection of distinct objects
\vspace{0.4cm}
\item \textbf{Element}: an object consisting a set
\end{itemize}
\end{frame}

\begin{frame}
\frametitle{Set: Notation}
\begin{itemize}
\item We write elements of a set within $\left\{\right\}$.
 \begin{itemize}
 \item $A = \left\{1, 2, 4, 6, 7\right\}$
 \end{itemize}
\vspace{0.4cm}
\item When we can write elements of a set using a general form, we use the notation $\left\{\mathrm{general \ form}|\mathrm{definition}\right\}$
 \begin{itemize}
 \item e.g., $B = \left\{x^2|x \ \mathrm{is \ an \ integer}, 1 \le x \le 4\right\}$
 \end{itemize}
\vspace{0.4cm}
\item If $x$ is an element of the set $A$, we write $x \in A$.
\vspace{0.4cm}
\item If $x$ is not an element of the set $A$, we write $x \notin A$.
\end{itemize}
\end{frame}

\begin{frame}
\frametitle{Set: Subset}
\begin{itemize}
\item If every element of $A$ is also in $B$, $A$ is called the \textbf{subset} of $B$, and denoted as $A \subseteq B$.
\vspace{0.4cm}
\item Example: Let $A = \left\{1, 2, 4, 6, 7\right\}$ and $B = \left\{1, 2, 7\right\}$. Then, $B \subseteq A$.
\end{itemize}
\end{frame}

\begin{frame}
\frametitle{Set Operation}
\begin{itemize}
\item \textbf{Intersection}: the intersection of $A$ and $B$, denoted as $A \cap B$, is the set of common elements to both sets. 
\vspace{0.4cm}
\item \textbf{Union}: the union of $A$ and $B$, denoted as $A \cup B$, is the set of all elements contained in either set.
\vspace{0.4cm}
\item \textbf{Complement}: the complement of $A$, denoted as $A^{c}$, is the set that contains elements that are not contained in $A$.
\vspace{0.4cm}
\item \textbf{Difference}: the difference of sets $A$ and $B$, denoted as $A/B$, is the set of elements in $A$ but not in $B$.
 \begin{itemize}
 \item $A/B$ is the same as $A \cap B^{c}$.
 \item $A/B$ is generally different from $B/A$.
 \end{itemize} 
\end{itemize}
\end{frame}

\begin{frame}
\frametitle{Set: Number System}
\begin{itemize}
\item \textbf{Natural numbers} ($\N$): set of 0 and positive integers ($\left\{0, 1, 2, \cdots \right\}$) 
\vspace{0.4cm}
\item \textbf{Integers} ($\Z$): set of all integers ($\left\{\cdots, -2, -1, 0, 1, 2, \cdots \right\}$)
\vspace{0.4cm}
\item \textbf{Rational numbers} ($\Q$): set of all numbers which can be represented as the fraction of integers
\vspace{0.4cm}
\item \textbf{Real numbers} ($\R$): set of all numbers whose squares are larger than or equal to 0.
 \begin{itemize}
 \item Real numbers which cannot be represented as the fraction of integers (e.g., $\sqrt{2}$) are called the \textbf{irrational numbers}
 \end{itemize} 
\vspace{0.4cm}
\item Using the set notation, $\N \subseteq \Z \subseteq \Q \subseteq \R$.
\end{itemize}
\end{frame}

\begin{frame}
\frametitle{Set: Additional Notations/Terms}
\begin{itemize}
\item \textbf{Empty set}: a set with no elements, denoted as $\emptyset$
\vspace{0.4cm}
\item If sets $A$ and $B$ have no elements in common (i.e., $A \cap B = \emptyset$), we say $A$ and $B$ are \textbf{disjoint}
\vspace{0.4cm}
\item We denote the union and the intersection of a sequence of sets ($A_1, A_2, \cdots, A_n$) as follows.
 \begin{eqnarray}
 \bigcap_{i = 1}^{n} A_i &=& A_1 \cap A_2 \cap \cdots \cap A_n \notag \\
 \bigcup_{i = 1}^{n} A_i &=& A_1 \cup A_2 \cup \cdots \cup A_n \notag 
 \end{eqnarray}
\end{itemize}
\end{frame}

\begin{frame}
\frametitle{Set: Open/Closed Intervals}
\begin{itemize}
\item \textbf{Interval}: a set of real numbers that the any numbers between the two points in the set are also included in the set.
\vspace{0.4cm}
\item \textbf{Open interval} is an interval that does not include the end points as elements. e.g., $A = \left\{a < i < b| i \in \R\right\}$
 \begin{itemize}
 \item We denote open intervals using parentheses $()$.
 \item e.g., $A = (a, b)$
 \end{itemize}
\vspace{0.4cm}
\item \textbf{Closed interval} is an interval that include the end points as elements. e.g., $A = \left\{a \leq i \leq b| i \in \R\right\}$
 \begin{itemize}
 \item We denote closed intervals using square brackets $[]$.
 \item e.g., $A = [a, b]$
 \end{itemize}
\end{itemize}
\end{frame}

\begin{frame}
\frametitle{Set: Exercises}
\begin{enumerate}
\item List all the elements of the set $A = \left\{x^3|x \in \Z, -2 \le x \le 2\right\}$
\vspace{0.4cm}
\item Let $A = \left\{-3, 1, 4, 6, 8, 13\right\}$ and $B = \left\{-5, -3, 1, 8, 11, 13\right\}$. Then what is
 \begin{enumerate}
 \item $A \cap B$
 \item $A \cup B$
 \item $A/B$
 \item $B/A$
 \end{enumerate}
\end{enumerate}
\end{frame}

\begin{frame}
\frametitle{Variable/Constant}
\begin{itemize}
\item \textbf{Variable}: a symbol which represent an arbitrary number
 \begin{itemize}
 \item value of a variable is not fully specified, so it can change/vary
 \end{itemize}
\vspace{0.4cm}
\item \textbf{Constant}: a quantity whose value do not change 
\end{itemize}
\end{frame}

\begin{frame}
\frametitle{Basic Number Properties}
\begin{itemize}
\item \textbf{Commutative Property}: the order of addition/multiplication does not affect the outcome
 \begin{eqnarray}
 a + b &=& b + a \notag \\
 a \times b &=& b \times a \notag
 \end{eqnarray}
\item \textbf{Associative Property}: the order of addition/multiplication does not matter as long as the sequence of operation is not changed
 \begin{eqnarray}
 (a + b) + c &=& a + (b + c) \notag \\
 (a \times b) \times c &=& a \times (b \times c) \notag
 \end{eqnarray}
\item \textbf{Distributive Property}: distribution of multiplication over addition/subtraction
 \begin{equation}
 a (b + c) = ab + ac \notag
 \end{equation}
\end{itemize}
\end{frame}

\begin{frame}
\frametitle{Expansion}
\begin{itemize}
\item Remove parentheses $()$ from the product of polynomials.
\vspace{0.4cm}
\item Monomial/Polynomial
 \begin{itemize}
 \item \textbf{Monomial}: product of a constant and a variable raised to some value, which looks like $ax^k$
  \begin{itemize}
  \item The constant ($a$) is called the \textbf{coefficient}, and the number of exponent ($k$) is called the \textbf{degree}/\textbf{order} of the monomial
  \end{itemize} 
 \item \textbf{Polynomial}: sum of monomials
  \begin{itemize}
  \item e.g., $2x + 3$, $x^3 - 4x + 5$
  \item The degree of a polynomial is the highest degree of its monomials 
  \end{itemize} 
 \end{itemize}
\item How to perform expansion?
 \begin{itemize}
 \item Repeatedly apply the distribution rule!
 \end{itemize}
\end{itemize}
\end{frame}

\begin{frame}
\frametitle{Expansion (cont.)}
\begin{itemize}
\item Question: Multiply out $(2x + 3)(x^2 - 8x + 5)$
\vspace{0.4cm}
 \begin{itemize}
 \item Answer:
 \begin{eqnarray}
 && (2x + 3)(x^2 - 8x + 5) \notag \\
 &=& 2x(x^2 - 8x + 5) + 3(x^2 - 8x + 5) \notag \\
 &=& 2x^3 - 16x^2 + 10x + 3x^2 - 24x + 15 \notag \\
 &=& 2x^3 - 13x^2 - 14x + 15 \notag
 \end{eqnarray}
 \end{itemize}
\end{itemize}
\end{frame}

\begin{frame}
\frametitle{Expansion (cont.)}
\begin{itemize}
\item Question: Multiply out $(x + 2)^3$
\vspace{0.4cm}
 \begin{itemize}
 \item Answer:
 \begin{eqnarray}
 && (x + 2)^3 \notag \\
 &=& (x + 2)(x + 2)^2 \notag \\
 &=& (x + 2)\left\{x(x + 2) + 2(x + 2)\right\} \notag \\
 &=& (x + 2)(x^2 + 4x + 4) \notag \\
 &=& x(x^2 + 4x + 4) + 2(x^2 + 4x + 4) \notag \\
 &=& x^3 + 6x^2 + 12x + 8 \notag
 \end{eqnarray}
 \end{itemize}
\end{itemize}
\end{frame}

\begin{frame}
\frametitle{Factoring}
\begin{itemize}
\item \textbf{Factoring}: writing a polynomial as a product of polynomials of lower degrees.
\vspace{0.4cm}
\item Factoring is the reverse of expanding parentheses.
\vspace{0.4cm}
\item (Factoring is a bit harder than expansion, and you need to get used to it...)
\vspace{0.4cm}
\item Tips
 \begin{itemize}
 \item If you find common factors, try grouping the terms containing them
  \begin{itemize}
  \item e.g., $x^3 - 5x^2 = x^2 \times x + x^2 \times (-5) = x^2(x - 5)$
  \end{itemize}
 \item Try applying the rules on the next page
 \end{itemize}
\end{itemize}
\end{frame}

\begin{frame}
\frametitle{Factoring/Expansion Rules}
\begin{itemize}
\item Some of the rules you often encounter:
 \begin{enumerate} 
 \item $(x + a)(x + b) = x^2 + (a + b)x + ab$
 \item $(ax + b)(cx + d) = acx^2 + (ad + bc)x + bd$
 \item $(x + a)^2 = x^2 + 2ax + a^2$
 \item $(x - a)^2 = x^2 - 2ax + a^2$
 \item $(x + a)(x^2 - ax + a^2) = x^3 + a^3$
 \item $(x - a)(x^2 + ax + a^2) = x^3 - a^3$
 \item $(x + y + z)^2 = x^2 + y^2 + z^2 + 2xy + 2yz + 2zx$
 \end{enumerate}
\vspace{0.4cm}
\item (Again, you don't neet to memorize them...)
\end{itemize}
\end{frame}

\begin{frame}
\frametitle{Factoring/Expansion: Exercises}
\begin{enumerate}
\item Multiply out:
 \begin{enumerate}
 \item $(x + 6)(x - 7) = $
 \item $(x^2 + 3)(x^2 + x + 7) = $
 \end{enumerate}
\vspace{0.4cm}
\item Factorize:
 \begin{enumerate}
 \item $x^2 - 2x - 3 = $
 \item $6x^2 - x - 1 = $
 \item $x^3 - 8 = $
 \item $x^3 - 3x^2 - 10x = $
 \end{enumerate}
\end{enumerate}
\end{frame}

\begin{frame}
\frametitle{Solving Equations/Inequalities}
\begin{itemize}
\item This is when we want find values of a variable satisfying an equation/range of values of a variable satisfying an inequality
\vspace{0.4cm}
\item Tips
 \begin{itemize}
 \item Rearrange terms so that the variable of interest is isolated
 \item Make sure to perform the same operations on both sides of equality/inequality
  \begin{itemize}
  \item Be careful when multiplying by negative numbers in solving inequalities!
  \end{itemize}
 \item Check answer
 \end{itemize}
\end{itemize}
\end{frame}

\begin{frame}
\frametitle{Solving Equations/Inequalities (cont.)}
\begin{itemize}
\item Question: Temperatures in Celsius (say $x$) is converted to those in Fahrenheit ($y$) using the following equation.
 \begin{equation}
 y = \frac{9}{5}x + 32 \notag
 \end{equation} 
Then,
 \begin{enumerate}
 \item Find the temperature when both are equal.
 \item When do temperatures measured in Fahrenheit are higher than those in Celsius?
 \end{enumerate}
\end{itemize}
\end{frame}

\begin{frame}
\frametitle{Solving Quadratics}
\begin{itemize}
\item After you rearrange the terms in the form $\mathrm{(i)} ax^2 + bx + c = 0 /\mathrm{(ii)} ax^2 + bx + c \leq 0 /\mathrm{(iii)} ax^2 + bx + c \geq 0$...
\vspace{0.4cm}
\item If you can easily factorize as $a(x + \alpha)(x + \beta)\ (\alpha > \beta, a > 0)$, then
 \begin{itemize}
 \item (i): $(x + \alpha) = 0$ and/or $(x + \beta) = 0 \Rightarrow x = -\alpha, -\beta$
 \item (ii): $(x + \alpha) \geq 0$ and  $(x + \beta) \leq 0 \Rightarrow -\alpha \leq x \leq -\beta$
 \item (iii): $(x + \alpha) \leq 0$ or $(x + \beta) \geq 0 \Rightarrow x \leq -\alpha, x \geq -\beta$
 \item How should we do when $a < 0$?
 \end{itemize}
\vspace{0.4cm}
\item When we cannot easily factorize... $\rightarrow$ \textbf{Complete the square}
 \begin{itemize}
 \item Transform the quadratic into the form $a(x \pm \alpha) = \beta$ and solve for $x$ by taking the square root of those terms.
 \end{itemize}
\end{itemize}
\end{frame}

\begin{frame}
\frametitle{Solving Quadratics: Complete the Square}
\begin{enumerate}
\item Move the constant to the RHS, and divide the equation by the coefficient on the squared term
 \begin{itemize}
 \item $x^2 + \frac{b}{a}x = -\frac{c}{a}$
 \end{itemize}
\vspace{0.2cm}
\item Divide the coefficient on x by 2, square the value, and add to both sides
 \begin{itemize}
 \item $x^2 + 2 \cdot \frac{b}{2a}x + \frac{b^2}{4a^2} = -\frac{c}{a} + \frac{b^2}{4a^2}$
 \end{itemize}
\vspace{0.2cm}
\item Factor the LHS into the form $(x \pm \alpha)^2$ and simplify the RHS
 \begin{itemize}
 \item $(x + \frac{b}{2a})^2 = \frac{b^2 - 4ac}{4a^2}$
 \end{itemize}
\vspace{0.2cm}
\item Take the square root of both sides
 \begin{itemize}
 \item $x + \frac{b}{2a} = \pm \sqrt{\frac{b^2 - 4ac}{4a^2}} = \pm \frac{\sqrt{b^2 - 4ac}}{2a}$
 \end{itemize}
\vspace{0.2cm}
\item Solve for $x$
 \begin{itemize}
 \item $x = \frac{-b \pm \sqrt{b^2 - 4ac}}{2a}$
 \end{itemize}
\end{enumerate}
\end{frame}

\begin{frame}
\frametitle{Solving Quadratics: Exercises}
\begin{itemize}
\item Solve the following equations/inequalities for $x$
 \begin{enumerate}
 \item $2x^2 - 4x - 3 = 0$
 \item $x^2 + x - 6 \leq 0$
 \item $x^3 = x^2 + 12x$
 \end{enumerate}
\end{itemize}
\end{frame}

\begin{frame}
\frametitle{Function}
\begin{itemize}
\item \textbf{Function} is a relation between sets which associate every element of the first set to exactly one element of the second set.
 \begin{itemize}
 \item describes how the values of the LHS variable (inputs) changes with values of RHS variables (outputs)
 \item input variables are often called the \emph{inpdependent} variables
 \item output variable os often called the \emph{dependent} variable
 \end{itemize}
\vspace{0.4cm}
\item Domain, image, range
 \begin{itemize}
 \item \textbf{Domain}: set over which a function is defined
 \item \textbf{Image}: output value of the function 
 \item \textbf{Range}: set of the images of all elements of the domain
 \end{itemize}
\end{itemize}
\end{frame}

\begin{frame}
\frametitle{Function: Notation}
\begin{itemize}
\item Let $f$ be a function which relates every element $x \in X$ to exactly one element $y \in Y$. Then we denote the relationship as
\begin{equation}
y = f(x) \notag
\end{equation}
or 
\begin{equation}
f: X \rightarrow Y \notag
\end{equation}
 \begin{itemize}
 \item Other letters often used to denote functions: $g$, $h$, $u$, $v$...
 \end{itemize}
\end{itemize}
\end{frame}

\begin{frame}
\frametitle{Function: Examples}
\begin{itemize}
\item $f(x) = x + 4 \ (x \in \R)$
\vspace{0.4cm}
\item $f(x) = x^2 + 2 \ (x \in \R)$
 \begin{itemize}
 \item What is the range of this function?
 \end{itemize}
\vspace{0.4cm}
\item $f(x_1, x_2) = 2x_1 - x_2 + 7 \ (x_1, x_2 \in \R)$
 \begin{itemize}
 \item $f$ maps/associates every element $(x_1, x_2) \in X$ to exactly one element of $y \in Y$.
 \end{itemize}
\vspace{0.4cm}
\item $x^2 + y^2 = 1$ is not a function!
 \begin{itemize}
 \item e.g., when $x = 1$, $y = \pm 1$
 \end{itemize}
\end{itemize}
\end{frame}

\begin{frame}
\frametitle{Graphing Functions}
\begin{itemize}
\item Why do you want to graph a function?
 \begin{itemize}
 \item Is it increasing/decreasing?
 \item How fast is it increasing/decreasing?
 \item ...
 \end{itemize}
\vspace{0.4cm}
\item Intersection of algebra and geometry
\end{itemize}
\end{frame}

\begin{frame}
\frametitle{Graphing Functions (cont.)}
\begin{columns}
\begin{column}{0.5\textwidth}
\includegraphics[scale = 0.4]{fig1_graph.png}
\end{column}
\begin{column}{0.5\textwidth}
\begin{itemize}
\item Graph of $y = f(x)$ is the collection of points satifying the relationship.
\vspace{0.4cm}
\item Compute the points $(x_0, f(x_0))$ for multiple $x_0$ and connect them!
\end{itemize}
\end{column}
\end{columns}
\end{frame}

\begin{frame}
\frametitle{Graphing Functions (cont.)}
\begin{columns}
\begin{column}{0.5\textwidth}
\includegraphics[scale = 0.4]{fig1_graph_1.png}
\end{column}
\begin{column}{0.5\textwidth}
\begin{itemize}
\item Areas above the curve (areas in blue): set of points satisfying the relationships $y > f(x)$ 
\vspace{0.4cm}
\item Areas below the curve (areas in green): set of points satisfying the relationships $y < f(x)$ 
\end{itemize}
\end{column}
\end{columns}
\end{frame}

\begin{frame}
\frametitle{Graphing Functions (cont.)}
\begin{itemize}
\item Linear function (polynomial of degree 1): $y = a + bx$
 \begin{itemize}
 \item Must pass the point $(0, a)$. 
 \end{itemize}
\vspace{0.4cm}
\item Quadratic function
 \begin{itemize}
 \item If we can factorize to $a(x + b)(x + c) \rightarrow$ must passt the points $(-b, 0)$ and $(-c, 0)$
 \item Complete the square and get $a(x \pm b)^2 + c \rightarrow$ must pass the point $(-b, c)$
  \begin{itemize}
  \item If $a > 0$: $(-b, c)$ is the global minimum
  \item If $a < 0$: $(-b, c)$ is the global maximum
  \end{itemize}
 \end{itemize}
\end{itemize}
\end{frame}

\begin{frame}
\frametitle{Graphing Functions: Exercises}
\begin{itemize}
\item Graph the following relationships.
 \begin{enumerate}
 \item $y = x^2 - 2x + 5$
 \item $y = 2x^2 + 2x - 12$
 \item $y \geq \frac{3}{2}x - 4$
 \end{enumerate}
\end{itemize}
\end{frame}

\begin{frame}
\frametitle{Functions Often Used in Social Science}
\begin{itemize}
\item Exponential function
\vspace{0.4cm}
\item Logarithmic function
\vspace{0.4cm}
\item (Trigonometric function)
\end{itemize}
\end{frame}

\begin{frame}
\frametitle{Exponential Function}
\begin{itemize}
\item An exponential function of $x$ is some constant (say $a$) raised to $x$: $f(x) = a^x$
 \begin{itemize}
 \item $f(x) = 4^x$
 \item $f(x) = (-2)^x$
 \item What is the range of an exponential function?
 \end{itemize}
\vspace{0.4cm}
\item Euler's constant $e$
 \begin{itemize}
 \item $e = \lim_{x \to \infty}(1 + \frac{1}{h})^h = 2.71828...$
 \item $e^x = \exp(x)$
 \end{itemize}
\end{itemize}
\end{frame}

\begin{frame}
\frametitle{Exponential Function: Basic Rules}
\begin{enumerate}
\item $a^{x_1} \times a^{x_2} = a^{x_1 + x_2} \ (x_1, x_2 \in \R)$
\item $(a^{x_1})^{x_2} = a^{x_1 x_2}$
\item $a^0 = 1$
\item $a^{-x} = \frac{1}{a^{x}}$
\item $a^{\frac{1}{x}} = \sqrt[x]{a}$
\end{enumerate}
\end{frame}

\begin{frame}
\frametitle{Logarithmic Funtion}
\begin{itemize}
\item Logarithm is the inverse of an exponential function. Logarithm of $x$ to base $a$ is the number such that $a$ power of it equals to $x$. Formally, 
\begin{equation}
y = \log_{a} x \Leftrightarrow x = a^{y} \notag
\end{equation}
\item Log of $x$ to base $e$ is called the \textbf{natural log}, and often denoted as 
\begin{equation}
\log_{e} x = \log x = \mathrm{ln}(x) \notag
\end{equation}
\item Question: what is the domain of a log function?
\end{itemize}
\end{frame}

\begin{frame}
\frametitle{Logarithmic Funtion: Basic Rules}
\begin{enumerate}
\item $\log_{a} x^{b} = b\log_{a} x$
\item $\log_{a} \frac{1}{x} = \log_{a} x^{-1} = -\log_{a} x$
\item $\log_{a} xy = \log_{a} x + \log_{a} y$
\item $\log_{a} \frac{x}{y} = \log_{a} (x \times y^{-1}) = \log_{a} x - \log_{a} y$
\item $\log_{x} x = 1$
\item $\log_{a} 1 = \log_{a} a^{0} = 0$
\item (Change of base) $\log_{b} x = \frac{\log_{a} x}{\log_{a} b}$
\end{enumerate}
\end{frame}

\begin{frame}
\frametitle{Exponential and Logarithmic Functions: Exercises}
\begin{enumerate}
\item Simplify the following expressions.
 \begin{enumerate}
 \item $2^{5} \times 4^{\frac{3}{2}} = $
 \item $2\log x - \log(3x) = $
 \item $a^{log_{a} x} = $
 \end{enumerate}
\vspace{0.4cm}
\item Derive the change of base formula. (Hint: by definition, $x$ is represented as...)
\end{enumerate}
\end{frame}

\begin{frame}
\frametitle{Why Do People Use Logs?}
\begin{itemize}
\item Log function is (one of) the most frequently used function(s) in political science.
\vspace{0.4cm}
\item But why?
\vspace{0.4cm}
\item Log function stretches out the scale when the value of the input is small while compresses the scale when it's large.
 \begin{itemize}
 \item Dealing with skewness of the data
 \item Express nonlinear relationship
 \item ...
 \end{itemize}
\end{itemize}
\end{frame}

\begin{frame}
\frametitle{Why Do People Use Logs? (cont.)}
\centering
\only<1>{\includegraphics[scale = 0.45]{fig1_log.pdf}}
\only<2>{\includegraphics[scale = 0.45]{fig1_log1.pdf}}
\end{frame}

\begin{frame}
\frametitle{Summation and Product Operators}
\begin{itemize}
\item Assume a sequence of variables/values $x_m, x_{m + 1}, \cdots, x_n \ (m < n)$
\vspace{0.4cm}
\item We denote their sum as follows.
 \begin{equation}
 \sum_{i = m}^{n} x_i = x_m + x_{m + 1} + \cdots + x_n \notag
 \end{equation}
\item We denote their product as follows.
 \begin{equation}
 \prod_{i = m}^{n} x_i = x_m \times x_{m + 1} \times \cdots \times x_n \notag 
 \end{equation}
\end{itemize}
\end{frame}

\begin{frame}
\frametitle{Summation and Product Operators: Examples}
\begin{itemize}
\item $\sum_{i = 1}^{10} i = 1 + 2 + \cdots + 10 = 55.$
\vspace{0.4cm}
\item $\prod_{i = 1}^{7} i = 7! = 840.$
\vspace{0.4cm}
\item $\sum_{i = 1}^{6} x_i = x_1 + x_2 + x_3 + x_4 + x_5 + x_6.$
\vspace{0.4cm}
\item $\prod_{j = 2}^{5} x_j = x_2 \times x_3 \times x_4 \times x_5.$
\end{itemize}
\end{frame}

\begin{frame}
\frametitle{Summation and Product Operators: Exercises}
\begin{enumerate}
\item Perform the following calculation.
 \begin{enumerate}
 \item $\sum_{i = 3}^{7} i^2$
 \item $\sum_{i = 1}^{n} a$
 \end{enumerate}
\vspace{0.4cm}
\item Show that
 \begin{itemize}
 \item $\sum_{i = 1}^{n} (x_i + y_i) = \sum_i^n x_i + \sum_i^n y_i$
 \item $\sum_{i = 1}^{n} (ax_i + b) = a\sum_{i = 1}^{n} x_i + nb$
 \end{itemize}
\end{enumerate}
\end{frame}

\begin{frame}
\frametitle{Tomorrow}
\begin{itemize}
\item Prioblem set 1 $\rightarrow$ review in the morning
\vspace{0.4cm}
\item Tomorrow
 \begin{itemize}
 \item Limits
 \item Derivative
 \item Unconstrained optimization
 \item Moore \& Siegel, Chapters 4-6 \& 8
 \end{itemize}
\end{itemize}
\end{frame}

\end{document}